\documentclass[../cheatSheetAlgoritmi.tex]{subfiles}
\begin{document}

\section{ASD8}
\textbf{Il gioco del Mikado/Shangai}
\begin{lstlisting}[caption=Posso vincere Mikado/Shangai in un unico turno?]
GRAPH buildGraph(int[] X, int[] Y, int n, int m)
	GRAPH G = new GRAPH
  	for i = 1 to n do
    	G.insertNode(i)
  	for i = 1 to m do
    	G.insertEdge(X[i], Y[i])
  	return G

boolean isShangaiSolvable(int[] X, int[] n, int m)
	GRAPH G = buildGraph(X, Y, n, m)
  	int time = 1
  	int[] dt = new int[1...G.size()]
  	int[] ft = new int[1...G.size()]
  	foreach u $\in$ G.V() do
    	dt[u] = 0
    	ft[u] = 0
  	foreach u $\in$ G.V() do
    	if dt[u] == 0 and hasCycle(G, u, time, dt, ft) then
      		return false
  	return true
\end{lstlisting}
Il problema può essere visto come un problema di assenza di cicli in un grafo. Dove i numeri da $1$ a \textit{n} che identificano i bastoncini sono difatto dei nodi, mentre invece le relazioni identificate dal verbo "sovrasta" degli archi. É molto semplice dunque realizzare un grafo con i dati che ci vengono forniti (si noti che $X[i]$ sovrasta \textit{Y[i]}) e verficare se esso è un \textit{DAG}, ovvero se non presenta cicli. 

Per farlo possiamo semplicemente avvalerci della funzione hasCycle per i grafi orientati, che si trova nella sezione: 8.5 Grafo orientato aciclico(DAG): DFS.

\bigskip

\textbf{Votazioni}
\begin{lstlisting}[caption=Le votazioni]
int[] election(GRAPH G, int[] status)
	QUEUE qpp = new QUEUE
  	QUEUE qpe = new QUEUE
  	int[] distanzePP = new int[1..G.size()]
  	int[] distanzePE = new int[1..G.size()]
  	foreach u $\in$ G.V() do
    	if status[u] == PP then
      		distanzePP[u] = 0
      		distanzePE[u] = $+\infty$
      		qpp.enqueue(u)
    	else if status[u] == PE then
      		distanzePP[u] = $+\infty$
      		distanzePE[u] = 0
      		qpe.enqueue(u)
    	else
      		distanzePP[u] = $+\infty$
      		distanzePE[u] = $+\infty$

  	while not qpp.isEmpty() then
    	NODE u = qpp.pop()
    	foreach v $\in$ G.adj(u) do
      		if distanzePP[v] == $+\infty$ then
        		distanzePP[v] = distanzePP[u] + 1
        		qpp.enqueue(v)
  
  	while not qpe.isEmpty() then
    	NODE u = qpe.pop()
    	foreach v $\in$ G.adj(u) do
      		if distanzePE[v] == $+\infty$ then
        		distanzePE[v] = distanzePE[u] + 1
        		qpe.enqueue(v)
  
  	int indecisi = 0
  	foreach u $\in$ G.V() do
    	if distanzePP[u] == distanzePE[u] then
      		indecisi = indecisi + 1
  
  	return indecisi
\end{lstlisting}
Per risolvere questo problema sono necessarie due visite in bfs assegnando dunque le distanze minori ai nodi dai nodi facenti parte del partito PE e una visita assegnando la distanza minima dai nodi PP. La visita viene effettuata incodando i corrispettivi membri del partito nella loro appostita coda con distanza pari a 0, la distanza viene aggiornata non appena vengono scoperti (lo stesso principio di ragionamento è presente nell'algoritmo di \textit{Erdos}). Una volta che abbiamo le distanze minime dai membri del partito PE e PP controlliamo se qualche nodo possiede la stessa dai membri dei due schieramenti, se questo avviene incrementiamo il numero degli indecisi.

Ritorniamo infine gli indecisi totali.

\end{document}