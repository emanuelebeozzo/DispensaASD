\documentclass[../cheatSheetAlgoritmi.tex]{subfiles}
\begin{document}

\subsection{ASD5}
\textbf{Grafo etichettato}
\begin{lstlisting}[caption=Il grafo contiene la sottosequenza LODE?]
boolean checkPath(GRAPH G, char[] l, char[] A, int n)
	foreach u $\in$ G.V() do
    	if checkLabels(G, l, A, n,1, u) then
      		return true

boolean checkLabels(GRAPH G, char[] l, char[] A, int n, int i, int u)
	if l[u] = A[i] then
    	if i = n then
      		return true
    	foreach v $\in$ G.adj(u) do
      		if checkLabels(G, l, A, n, i+1, v) then
        		return true
  	return false
\end{lstlisting}
Questo algoritmo per risolvere il problema effettua una visita a
partire da ogni nodo, verificando nodo per nodo se la sequenza viene rispettata, altrimenti, interrompe la
visita qualora le etichette non corrispondano. Semplicemente viene fatto passare per ogni nodo l'algoritmo \textit{checkLabels}, il quale in dfs controlla se il suo valore è pari al valore del vettore nella posizione passata, se si allora esso propaga l'algoritmo con la posizione successiva agli adiacenti, altrimenti non fa nulla. Qualora si arrivasse alla fine del percorso l'algoritmo ritorna true. Si noti che la complessità è superpolinomiale, in quanto è
necessario seguire tutti i possibili percorsi, anche tenendo conto che il
grafo e aciclico.

\textbf{Griglia quadra}

Soluzione con costruzione del grafo:
\begin{lstlisting}[caption=Griglia quadra - costruzione grafo]
(GRAPH, int[]) costruisciGrafo(int[][] A, int n)
	GRAPH G = new GRAPH
  	int[] colors = new colors[1..n]
  	% inserimento nodi
  	for i = 1 to n $\cdot$ n do
    	G.insertNode(i)
  	% inserimento archi
  	for riga = 1 to n do
    	for colonna = 1 to n do
      		% id sequenziale del nodo
      		int idNodo = colonna + (riga - 1) $\cdot$ n
      		colors[idNodo] = A[riga][colonna]
      		int idAdiacente
      		if riga $\neq$ 1 then 
        		idAdiacente = colonna + (riga - 2) $\cdot$ n
        		G.insertEdge(idNodo, idAdiacente)
      		if colonna $\neq$ 1 then
        		idAdiacente = (colonna - 1) + (riga - 1) $\cdot$ n
        		G.insertEdge(idNodo, idAdiacente)
      		if colonna $\neq$ n then
        		idAdiacente = (colonna + 1) + (riga - 1) $\cdot$ n
        		G.insertEdge(idNodo, idAdiacente)
      		if riga $\neq$ n then
        		idAdiacente = colonna + riga $\cdot$ n
        		G.insertEdge(idNodo, idAdiacente)
  	return (G, colors)

int distanza1_3(int[][] A, int n)
	GRAPH G
  	int[] colors = new int[1..n$\cdot$n]
  	(G, colors) = costruisciGrafo(A, n)
  	QUEUE q = new QUEUE
  	int[] distanze = new distanze[1..G.size()]
  	foreach u $\in$ G.V() do
    	if colors[u] == 1 then
      		distanze[u] = 0
      		q.enqueue(u)
    	else
      		distanze[u] = $+\infty$
  	while not q.isEmpty() do
    	NODE u = q.dequeue()
    	foreach v $\in$ G.adj(u) do
      		if distanze[v] == $+\infty$ then
        		distanze[v] = distanze[u] + 1
        		if colors[v] == 3 then 
          			return distanze[v]
        		else
          			q.enqueue(v)
  	return $+\infty$
\end{lstlisting}
La costruzione del grafo partendo dalla matrice ha un costo di $\Theta(n^2)$, e si può fare con dei semplici accorgimenti: un assegnazione univoca di un identificatore del nodo, semplicemente un numero da 1 a $n^2$ e un controllo sugli indici per non selezionare un nodo non esistente. 

Questo algoritmo costruisce un grafo partendo da una matrice, si occupa inoltre di inserire all'interno di un vettore \textit{colors[i]} il valore della cella della matrice associata al nodo \textit{i}. \\ Avendo questo è semplice trovare la distanza minima dalle celle di colore 1 ad una cella di colore 3, effettuando una bfs assegnando distanza 0 dei nodi con colore pari a 1 e procedendo come insegna \textit{Erdos}. Ritorniamo la distanza della prima occorrenza di colore 3 in quanto essendo la bfs una visita per livelli, la prima occorrenza è anche la meno distante.

\bigskip

Soluzione con bfs direttamente sulla matrice:
\begin{lstlisting}[caption=Griglia quadra]
int grid(int[][] M, int n)
  int[] dr = [-1, 0, +1, 0] % Mosse possibili sulle righe
  int[] dc = [0, -1, 0, +1] % Mosse possibili sulle colonne
  int[] distance = new int[1...n][1...n]
  QUEUE Q = QUEUE
  for r = 1 to n do
    for c = 1 to n do
      distance[r][c] = iif(M[r][c] == 1, 0, -1)
      if M[r][c] == 1 then
        Q.enqueue(<r, c>)
  while not Q.isEmpty() do
    int, int r, c = Q.dequeue()   % Riga, colonna della cella corrente
    for i = 1 to 4 do
      nr = r + dr[i] % Nuova riga
      nc = c + dc[i] % Nuova colonna
      if 1 $\leq$ nr $\leq$ n and 1 $\leq$ nc $\leq$ n and distance[nr][nc] < 0 then
        distance[nr][nc] = distance[r][c] + 1
        if M[nr][nc] == 3 then
          return distance[nr][nc]
        else
          Q.enqueue(<nr, nc>)
\end{lstlisting}
Il principio di funzinamento è sempre lo stesso, ovvero una visita in bfs assegnando ai nodi di colore 1 la distanza 0. Il foreach della bfs viene fatto corrispondere ad un for sulle mosse possibili. Viene inoltre controllato se viene sforato uno dei possibili indici, se non viene sforato allora ci si comporta come nell'algoritmo precedente.

\end{document}