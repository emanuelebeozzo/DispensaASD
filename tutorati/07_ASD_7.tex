\documentclass[../cheatSheetAlgoritmi.tex]{subfiles}
\begin{document}

\section{ASD7}
\textbf{Colorazione Alberi}

Il problema prevede che ogni nodo abbia un colore differente rispetto a quelli adiacenti e quindi non bisogna lasciarsi ingannare dal fatto che vi siano a disposizione un numero di colori pari al numero di nodi presenti all'interno dell'albero. Si può anche facilmente intuire che un albero è un grafo bipartito in quanto possiamo suddividere i nodi in base a livelli pari e dispari; per questo motivo è bi-colorabile e dunque possiamo usare solamente 2 colori per poter colorare completamente l'albero e, per ottenere il costo minore, dobbiamo usare per forza i colori 1 e 2. L'ultima cosa di cui non possiamo essere certi è con che colore cominciare: visto che si tratta di un albero vorremmo che i livelli contenenti più nodi siano colorati con il colore 1 mentre quelli contenenti il numero minore con il colore 2.

Di seguito viene proposta un'implementazione più intelligente che effettua una sola visita dell'albero salvando i nodi in 2 insiemi differenti e assegnando a quello con il numero maggiore il colore 1 mentre all'altro il colore 2; ovviamente viene eseguita una BFS utilizzando una coda.
\begin{lstlisting}[caption=Colorazione Alberi]
int minColoring(TREE T)
	if T = nil then
    	return 0
  	else
    	QUEUE q = new QUEUE
    	q.enqueue(T)
    	SET s1 = new SET
    	SET s2 = new SET
    	int nFigli = q.size()
    	bool first = true
    	while not q.isEmpty() do
      		TREE extracted = q.dequeue()
      		if first then
        		s1.insert(extracted)
      		else
        		s2.insert(extracted)
      		nFigli = nFigli - 1
      		if extracted.left $\neq$ nil then
        		q.enqueue(extracted.left)
      		if extracted.right $\neq$ nil then
       			q.enqueue(extracted.right)
      		if nFigli $==$ 0 then
        		first = not first
        		nFigli = q.size()
    	if s1.size() > s2.size() then
      		return s1.size() + 2*s2.size()
    	else
      		return s2.size() + 2*s1.size()
\end{lstlisting}
Invece ecco l'algorirtmo proposto da montresor, sostanzialmente si prova a colorare il grafo in 2 modi (cioè in un caso colorando la radice con il colore 1 mentre nell'altro colorando la radice con il colore 2) e si tiene il risultato che costa di meno.
\begin{lstlisting}[caption=Colorazione Alberi (by Montresor)]
int countRec(TREE T, int color)
	if T $==$ nil then
    	return 0
  	else
    	return color + countRec(T.left, 3 - color) + countRec(T.right, 3 - color)

int minColoring(TREE T)
	return min(countRec(T, 1), countRec(T, 2))
\end{lstlisting}

\newpage
\end{document}
