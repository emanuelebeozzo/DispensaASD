\documentclass[../cheatSheetAlgoritmi.tex]{subfiles}
\begin{document}

\subsection{ASD3}
\textbf{É un albero red-black?}\\
\begin{lstlisting}[caption=É un albero red-black?]
boolean isRedBlack(TREE T)
	% controllo sulla radice
  	if T.color == RED then
    	return false
  	else
    	return (blackHeight(T) > 0)

int blackHeight(TREE T)
  	% Tutte le foglie devono essere rosse
  	if T.value == nil then
    	return iif(T.color == RED, -1, 1)
  	% Ogni nodo rosso deve avere entrambi i figli neri
  	if T.color == RED and T.parent $\neq$ nil and T.parent.color == RED then
    	return -1
  	% Il numero di nodi neri incontrati su tutti i possibili percorsi dalla radice ad una foglia deve essere lo stesso
  	int bhL = blackHeight(T.left)
  	int bhR = blackHeight(T.right)
  	if bhL < 0 or bhR < 0 or bhL $\neq$ bhR then
    	return -1
  	else
    	return bhL + iif(t.color == BLACK, 1, 0)
\end{lstlisting}
Per verificare se un albero è red-black è necessario assicurarsi che tutti i vincoli siano rispettati, in particolare che nessuno di essi venga violato:
\begin{itemize}
 	\item La radice deve essere nera
 	\item Tutte le foglie sono nere
 	\item Entrambi i figli di un nodo rosso sono neri
 	\item Tutti i cammini semplici da un nodo \textit{u} ad una delle foglie contenute nel sottoalbero radicato in \textit{u} hanno lo stesso numero di nodi neri.\\ Una volta ricevuto l'albero in input l'algoritmo ne controlla la radice, restituendo false se il vincolo viene violato. A questo punto l'algoritmo si appoggia su una procedura ricorsiva che restituisce iun intero (necessario per il vincolo 4, ovvero quello dei cammini).
Ogni volta che la parte ricorsiva si porta in una foglia (il valore è nil) allora controlla che il suo colore sia corretto, se si allora ritorna 1 (in quanto è un nodo nero), -1 altrimenti. Viene verificato il vincolo 3 mediante un controllo sul padre qualora il colore della foglia esaminata sia rossa. Controlla infine le altezze nere, se non combaciano, oppure un vincolo è stato violato nei livelli sottostanti (il valore ottenuto è $<$ 0) esso torna -1. In questo modo è possibile risalire con un semplice controllo di maggiore se i vincoli sono stati rispettati o meno.
\end{itemize}\

\end{document}