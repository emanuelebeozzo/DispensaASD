\documentclass[../cheatSheetAlgoritmi.tex]{subfiles}
\begin{document}

\subsection{ASD10}
\textbf{Prima Occorrenza}

Visto che il vettore di interi che viene passato è ordinato possiamo utilizzare la ricerca binaria in modo da trovare in complessità logaritmica la prima occorrenza di un numero \textit{v}. Di seguito viene proposta un'implementazione definita poco elegante ma che ci è venuta subito in mente: l'idea è quella di continuare la ricerca dicotomica nella parte sinistra dell'array quando si trova l'elemento cercato (con l'accortezza di includere all'interno di questo array anche l'elemento appena analizzato - nel caso in cui fosse effettivamente la prima occorrenza) e restituire l'indice soltanto quando si analizza uno e un solo elemento.
\newpage
\begin{lstlisting}[caption=Prima Occorrenza]
int binaryFirst(int[] A, int start, int end, int v)
	if start $==$ end then
		return start
  	else
    	int m = (start + end)/2
    	if A[m] $==$ v then
      		return binaryFirst(A, start, m, v)
    	else if A[m] < v then
        	return binaryFirst(A, m+1, end, v)
      	else
        	return binaryFirst(A, start, m-1, v)

int first(int[] A, int n, int v)
	return binaryFirst(A, 1, n, v)
\end{lstlisting}
Nella soluzione seguente Montresor decide semplicemente di unificare i casi in un unico if else con la condizione del $\leq$
\begin{lstlisting}[caption=Prima Occorrenza (by Montresor)]
int firstRec(int[] A, int i, int j, int v)
	if i $==$ j then
    	return i
  	else
    	int m = (i + j)/2
    	if A[m] $\leq$ v then
      		return binaryFirst(A, i, m, v)
    	else
      		return binaryFirst(A, j, m+1, v)
      
int first(int[] A, int n, int v)
	return binaryFirst(A, 1, n, v)
\end{lstlisting}

\end{document}
