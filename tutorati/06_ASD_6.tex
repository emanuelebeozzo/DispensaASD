\documentclass[../cheatSheetAlgoritmi.tex]{subfiles}
\begin{document}

\section{ASD6}
\textbf{Nodi a distanza d}
\begin{lstlisting}[caption= Nodi che distano un numero di archi $\leq$ d]
int nodiMinUgualeDistanza(GRAPH G, NODE r, int d)
	int[] distanze = new int[1..G.size()]
  	NODE[] parents = new NODE[1..G.size()]
  	erdos(G, r, distanze, parents)
  	int nNodi = 0
  	foreach u $\in$ G.V() do
    	if distanze[u] $\leq$ d then 
      		nNodi = nNodi + 1 
  	return nNodi
\end{lstlisting}
Algoritmo molto semplice, una volta trovate tutte le distanze dei nodi dalla radice grazie ad $Erdos$ essa confronta le distanze con il valore d, se minore o uguale allora incrementa un contatore, che sarà ritornato alla fine dei confronti.

\bigskip

\textbf{Albero Allopato}

Un grafo e un "albero binario alloopato" se
deriva da un albero binario nel modo seguente: è presente un nodo del
grafo per ogni nodo dell’albero; e presente un arco per ogni coppia
\textit{(padre,figlio)}, orientato dal padre al figlio; e presente un arco da ogni
foglia alla radice dell’albero binario, chiudendo così un ciclo (loop). Gli
archi sono pesati da una funzione $W : E\rightarrow Z$; gli archi possono avere
peso negativo, ma non esistono cicli la cui somma dei pesi sia negativa.
\begin{lstlisting}[caption= Albero Binario Allopato]
int dfsSearch(GRAPH G, NODE x, NODE t, NODE r)
	if x == t then
    	return 0
  	int min = $+\infty$ 
  	foreach y $\in$ G.adj(x) do
    	if y $\neq$ r then
      		int d = dfsSearch(G, y, t, r)
      	if d + w(x, y) < min then
        	min = d + w(x, y)
  	return min
 
int computeDist(GRAPH G, NODE r, NODE u, NODE v)
	d = dfsSearch(G, u, v, r)
  	if d < +$\infty$ then
    	return d
  	else
  		return dfsSearch(G, u, r, -1) + dfsSearch(G, r, v, r)
\end{lstlisting}
Il compito di dfsSearch è quello di ricavare la distanza minima tra una foglia \textit{x} e la foglia \textit{t} con un controllo postordine.
Questa procedura tornerà un numero $n < +\infty$ se l'elemento t è nel sottalbero cercato, mentre $+\infty$ se non è presente
Si noti che quando viene propagata la visita in dfs non viene mai fatto per la radice. 

Proviamo quindi come primo approccio quello di chiamare \textit{dfsSearch} sui nodi \textit{u} e \textit{v} di cui si vuole la distanza minima, se il valore ritornto dalla funzione è $+\infty$ significa che il nodo \textit{v} non è un diretto discendente di \textit{u}, dunque devo necessariamente passare per la radice.
Detto questo, il costo effettivo della ricerca è dunque dato dalla distanza minima da u alla radice, sommato alla distanza minima dalla radice dalla radice al nodo v, ovviamente anche qui non bisogna passare ancora per la radice. Osserviamo che \textit{dfsSearch} funziona in questo caso, dato che sia \textit{u} che \textit{v} sono figli della radice dell'albero.

\end{document}
