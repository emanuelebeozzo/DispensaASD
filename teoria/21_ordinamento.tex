\documentclass[../cheatSheetAlgoritmi.tex]{subfiles}
\begin{document}

\chapter{Algoritmi di Ordinamento}
\section{Problema dell'ordinamento}
\textbf{Problema}: Data una sequenza $A = a_1, a_2 , ..., a_n$ di $n$ valori, restituire una sequenza $B = b_1 , b_2 , ..., b_n$ che sia una permutazione di $A$ e tale per cui $b_1 \leq b_2 \leq ... \leq b_n$.

Finora abbiamo studiato i seguenti algoritmi di ordinamento:

\begin{itemize}
	\item SelectionSort - $\Theta(n^{2})$
	\item InsertionSort - $\Omega(n), \mathcal{O}(n^{2})$
	\item ShellSort - $\Omega(n), \mathcal{O}(n^{3/2})$
	\item MergeSort - $\Theta(n \log n)$
	\item HeapSort - $\Theta(n \log n)$
	\item QuickSort - $\Omega(n), \mathcal{O}(n^{2})$
\end{itemize} 
\textcolor{red}{Tutti questi algoritmi sono basati su confronti}, cioè le decisioni sull'ordinamento vengono prese in base al confronto $(<, =, >)$ fra due valori e la diretta conseguenza è che gli algoritmi migliori hanno una complessità pari a \textcolor{red}{$\mathcal{O}(n \log n)$} (InsertionSort e ShellSort sono più performanti solo in casi particolari).\\
E' possibile dimostrare che \textcolor{red}{qualunque algoritmo di ordinamento basato su confronti} ha una complessità \textcolor{red}{$\Omega(n \log n)$}:
\begin{itemize}
	\item Consideriamo un qualunque algoritmo di ordinamento $A$ basato su confronti
	\item Assumiamo che tutti i valori siano distinti (non si perde di generalità con tale assunzione)
	\item L'algoritmo $A$ può essere rappresentato tramite un \textcolor{red}{albero di decisione}, un albero binario che rappresenta i confronti tra gli elementi
\end{itemize}
\textbf{Albero di decisione - Proprietà}
\begin{itemize}
	\item \textcolor{red}{cammino radice-foglia in un albero di decisione}: sequenza di confronti che vengono eseguiti dall'algoritmo corrispondente
	\item \textcolor{red}{Altezza dell'albero di decisione}: numero confronti che vengono eseguiti dall'algoritmo corrispondente nel caso pessimo
\end{itemize}

Si considerino tutti gli alberi di decisione ottenibili da algoritmi di ordinamento basati su confronti

\bigskip
\textbf{Lemma 1}: Un albero di decisione per l'ordinamento di $n$ elementi contiene almeno $n!$ foglie

\bigskip
\textbf{Lemma 2}: Sia $T$ un albero binario in cui ogni nodo interno ha esattamente 2 figli e sia $k$ il numero delle sue foglie. L'altezza dell'albero è almeno $\log k$ - ovvero $\Omega(\log k)$.

\bigskip
\textbf{Teorema}: Il numero di confronti necessari per ordinare $n$ elementi nel caso
peggiore è $\Omega(n \log n)$

\section{Algoritmi basati su confronti}
\subsection{Selection Sort}
\begin{lstlisting}[ caption= Selection Sort]
SelectionSort(ITEM[] A, int n)
	for i = 1 to n - 1 do
		int min_temp = min_from_to(A, i, n)
		A[min_temp] $\leftrightarrow$ A[i]

int min_from_to(ITEM[] A, int inizio, int fine)
	int i_min = inizio
  	for indice = inizio + 1 to fine do
		if S[i_min] > S[indice] then
			i_min = indice
	return i_min
\end{lstlisting}
\textbf{Costo computazionale}: La complessità nel caso medio, pessimo ed ottimo è uguale e vale $\mathcal{O}(n^{2})$.
\subsection{Insertion Sort}
\begin{lstlisting}[ caption= Insertion Sort]
insertionSort(ITEM[] A, int n)
	for i = 2 to n do
		% Salvo l'elemento che dovro' shiftare
		ITEM temp = A[i]
		% Mi salvo la sua posizione
		int j = i
		% Scorro il mazzo a ritroso trovando la sua posizione
		while j > 1 and temp < A[j-1] do
			A[j] = A[j-1]
			j = j - 1
		A[j] = temp
\end{lstlisting}
\textbf{Costo computazionale}: La complessità nel caso medio e pessimo  e vale $\mathcal{O}(n^{2})$, mentre per il caso ottimo la complessità è invece $\mathcal{O}(n)$. \\
Questo tipo di algoritmo è indicato per quando si vogliono riordinare piccoli insiemi di elementi.
\newpage
\subsection{Merge Sort}
\begin{lstlisting}[ caption= Merge Sort]
mergeSort(ITEM[] A, int n)
	mergeSort_rec(A, 1, n)

mergeSort_rec(ITEM[] A, int first, int last)
  	if first $\leq$ last do
		int mid = (first + last)/2
		mergeSort_rec(A, first, mid)
		mergeSort_rec(A, mid + 1, last)
		merge(A, first, mid, last)

merge(ITEM[] A, int first, int mid, int last)
  	int i, j, k, h
  	% vettore di appoggio
  	ITEM[] B = new ITEM[1..last]
  	i = first
  	j = mid + 1
  	k = first
  	% riempimento vettore di appoggio in maniera ordinata
  	while i $\leq$ mid and j $\leq$ last do
		if A[i] < A[j] then
			B[k] = A[i]
			i = i + 1
		else  
			B[k] = A[j]
			j = j + 1
		k = k + 1
	j = last
  	for h = mid downto i do
  		A[j] = A[h]
  		j = j - 1
  	for j = first to k - 1 do
  		A[j] = B[j]
\end{lstlisting}
\textbf{Costo computazionale}: La complessità di merge è di $\mathcal{O}(n)$, mentre la complessità di merge sort vale in tutti i casi $\mathcal{O}(n\log{}n)$.\
\subsection{Heap Sort}
\textbf{Riferimento}: Vedi il capitolo riguardante \hyperref[sec:heap]{\emph{Strutture dati speciali - Heap}}.
\subsection{Quick Sort}
\textbf{Riferimento}: Vedi il capitolo riguardante \hyperref[sec:quicksort]{\emph{Divide-et-Impera}}.

\section{Algoritmi NON basati su confronti}
\subsection{Spaghetti Sort}
\textbf{Algoritmo Spaghetti Sort - $\mathcal{O}(n)$}
\begin{enumerate}
	\item Prendi $n$ spaghetti
	\item Taglia lo spaghetto $i$-esimo in modo proporzionale all'$i$-esimo valore da ordinare
	\item Con la mano, afferra gli $n$ spaghetti e appoggiali verticalmente sul tavolo
	\item Prendi il più lungo, misuralo e metti il valore corrispondente in fondo al vettore da ordinare
	\item Ripeti (4) fino a quando non hai terminato gli spaghetti
\end{enumerate}
\subsection{Counting Sort}
\textbf{Assunzione}\\
I numeri da ordinare sono compresi in un intervallo $[1...k]$

\bigskip
\textbf{Come funziona}\\
Costruisce un array $B[1...k]$ che conta il numero di volte che un volre compreso in $[1...k]$ compare in $A$ e rialloca dunque i valori così ottenuti nel vettore da ordinare $A$.

\bigskip
\textbf{Miglioramenti}\\
L'intervallo non deve necessariamente iniziare in 1 e finire in $k$; qualunque intervallo di cui conosciamo gli estremi può essere utilizzato nel \emph{Counting Sort}
\begin{lstlisting}[caption=Counting Sort]
countingSort(int[] A, int n, int k)
	int[] B = new int[1...k]
	for i = 1 to k do
		B[i] = 0
	for j = 1 to n do
		B[A[j]] = B[A[j]] + 1
	int j = 1
	for i = 1 to k do
		while B[i] > 0 do
			A[j] = i
			j = j + 1
			B[i] = B[i] - 1
\end{lstlisting}
La complessità del Counting Sort è dunque pari a $\mathcal{O}(n + k)$: se $k$ è $\mathcal{O}(n)$, allora la complessità del Counting Sort è $\mathcal{O}(n)$.\\
\textbf{ATTENZIONE}: se $k$ è $\mathcal{O}(n^{3})$ questo algoritmo è il peggiore di tutti quelli visti fino ad ora!

\subsection{Pigeonhole Sort}
Cosa succede se i valori da ordinare non sono numeri interi ma ad esempio dei record associati ad una chiave che vanno ordinati? Non è possibile usare il counting ma si può invece utilizzare delle \emph{Liste Concatenate}

\subsection{Bucket Sort}
\textbf{Ipotesi sull'input}
\begin{itemize}
	\item Valori Reali uniformementi distribuiti nell'intervallo $[0, 1)$
	\item Qualunque insieme di valori distribuiti uniformemente può essere normalizzato nell'intervallo $[0, 1)$ in tempo lineare
\end{itemize}
\textbf{Idea}
\begin{itemize}
	\item Dividere l'intervallo in $n$ sottointervalli di dimensione $1/n$, detti \textcolor{red}{bucket}, e poi distribuire gli $n$ numeri nei bucket
	\item Per l'ipotesi di uniformità, il numero atteso di valori nei bucket è 1
	\item Possono essere ordinati con Insertion Sort
\end{itemize}

\section{Proprietà degli algoritmi di Ordinamento}
Un algoritmo di ordinamento viene detto \textcolor{red}{stabile} se preserva l'ordine iniziale tra due elementi con la stessa chiave. Esempi di algoritmi stabili sono Insertion Sort, Merge Sort e Pigeonhole Sort. Per rendere un qualunque algoritmo di ordinamento stabile è sufficiente usare come chiave di ordinamento la coppia (chiave, posizione iniziale).

\section{Tabella Riassuntiva}
\begin{itemize}
	\item \textbf{Insertion Sort}: $\Omega(n), \mathcal{O}(n^{2})$, stabile, sul posto, iterativo. Adatto per piccoli valori, sequenze quasi ordinate.
	\item \textbf{Merge Sort}: $\Theta(n \log n)$, stabile, richiede $\mathcal{O}(n)$ spazio aggiuntivo, ricorsivo (richiede $\mathcal{O}(\log n)$ spazio sullo stack). Buona performance in cache, buona parallelizzazione.
	\item \textbf{Heap Sort}: $\Theta(n \log n)$, non stabile, sul posto, iterativo. Cattiva performance in cache, cattiva parallelizzazione. Preferito in sistemi embedded.
	\item \textbf{Quick Sort}: $\mathcal{O}(n \log n)$ in media, $\mathcal{O}(n^{2})$ nel caso peggiore, non stabile, ricorsivo (richiede $\mathcal{O}(\log n)$ spazio sullo stack). Buona performance in cache, buona parallelizzazione, buoni fattori moltiplicativi.
	\item \textbf{Counting Sort}: $\Theta(n + k)$, richiede $\mathcal{O}(k)$ memoria aggiuntiva, iterativo. Molto veloce quando $k = \mathcal{O}(n)$.
	\item \textbf{Pigeonhole Sort}: $\Theta(n + k)$, stabile, richiede $\mathcal{O}(n + k)$ memoria aggiuntiva, iterativo. Molto veloce quando $k = \mathcal{O}(n)$.
	\item \textbf{Bucket Sort}: $\mathcal{O}(n)$ nel caso i valori siano distribuiti uniformemente, stabile, richiede $\mathcal{O}(n)$ spazio aggiuntivo.
	\item \textbf{Shell Sort}: $\mathcal{O}(n \sqrt{n})$, stabile, adatto per piccoli valori, sequenze quasi ordinate.
	\item \textbf{Tim Sort}:  $\Omega(n), \mathcal{O}(n \log n)$, algoritmo ibrido basato su Merge Sort e Insertion Sort che cerca sequenze consecutive (run) già ordinate.
\end{itemize}
\end{document}
