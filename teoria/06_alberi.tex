\documentclass[../cheatSheetAlgoritmi.tex]{subfiles}
\begin{document}

\section{Alberi}
\textbf{Descrizione}: Un albero radicato consiste in un insieme di nodi e un insieme di archi orientati che connettono coppie di nodi. Un nodo è definito come radice e, a differenza degli altri nodi, non ha archi entranti e esiste un percorso unico tra la radice e ogni nodo.

\subsection{Alberi binari}
Un albero binario è un albero radicato in cui ogni nodo ha al massimo due figli, identificati come figlio sinistro e figlio destro.\\
\textbf{Specifica}: Tutte le operazioni possibili sugli alberi binari sono le seguenti. 
\begin{lstlisting}[caption= Specifica albero binario]
% Costruisce un nuovo nodo, contenente v, senza figli o genitori
Tree(ITEM v)
% Legge il valore memorizzato nel nodo
ITEM read()
% Modifica il valore memorizzato nel nodo
write(ITEM v)
% Restituisce il padre, oppure nil se questo nodo e' radice
TREE parent()
% Restituisce il figlio sinistro (destro) di questo nodo o restituisce nil se assente
TREE left()
TREE right()
% Inserisce il sottoalbero radicato in t come figlio sinistro (destro) di questo nodo
insertLeft(TREE t)
insertRight(TREE t)
% Distrugge (ricorsivamente) il figlio sinistro (destro) di questo nodo
deleteLeft()
deleteRight()
\end{lstlisting}
\newpage
\begin{flushleft}
\textbf{Memorizzazione}: Ogni nodo dell'albero è memorizzato come sottoalbero. Per ogni nodo è necessario memorizzare 3 campi: parent, che è una reference al nodo padre, left che è una reference al nodo figlio sinistro e right che è una reference al figlio destro.
\end{flushleft}
\textbf{Implementazione}: Implementazione delle operazioni sugli alberi di ricerca, usando la specifica definita sopra. 
\begin{lstlisting}[caption= Implentazione alberi binari]
TREE
	ITEM V	
	TREE parent
	TREE left
	TREE right

Tree(Item v)
	TREE t = new TREE
	t.parent = nil
	t.left = t.right = nil
	t.value = v
	return t
	
insertLeft(TREE T)
	if left $==$ nil then
		T.parent = this
		left = T
		
insertRight(TREE T)
	if right $==$ nil then
		T.parent = this
		right = T
		
deleteLeft()
	if left $\neq$ nil then
		left.deleteLeft()
		left.deleteRight()
		left = nil
		
deleteRight()
	if right $\neq$ nil then
	right.deleteLeft()
	right.deleteRight()
	right = nil
\end{lstlisting}
\newpage

\subsubsection{Visite alberi binari}
\textbf{Problema}: Dato un albero T, visitare tutti i nodi della struttura.\\
\textbf{Soluzione}: Esistono principalmente due metodologie per visitare un albero:
\begin{itemize}
	\item DFS (Depth-First search - visita in profondità): per visitare un albero si visita ricorsivamente ognuno dei suoi sottoalberi. Tre varianti: pre/post/in order. Richiede uno stack.
\end{itemize}
\begin{lstlisting}[caption= Visita DFS albero bin]
preorder_dfs(TREE albero)
	if albero $\neq$ nil then
		print albero.read()
		preorder_dfs(albero.left())
		preorder_dfs(albero.right())

inorder_dfs(TREE albero)
	if albero $\neq$ nil then
		inorder_dfs(albero.left())
		print albero.read()
		inorder_dfs(albero.right())

postorder_dfs(TREE albero)
	if albero $\neq$ nil then
		postorder_dfs(albero.left())
		postorder_dfs(albero.right())
		print albero.read()
\end{lstlisting}
\begin{itemize}
 	\item BFS(Breadth-First search - visita in ampiezza): ogni livello dell'albero viene visitato, uno dopo l'altro partendo dalla radice. Richiede una coda. 
\end{itemize}
\begin{lstlisting}[ caption= Visita BFS albero binario]
bfs(TREE albero)
	QUEUE q = new QUEUE
	q.enqueue(albero)
	while not q.empty() do
		TREE tmp = q.dequeue()
		print tmp.read()
		if tmp.left() $\neq$ nil then
			q.enqueue(tmp.left())
		if tmp.right() $\neq$ nil then
			q.enqueue(tmp.right())
\end{lstlisting}
\newpage

\subsubsection{Esempi visite alberi binari con DFS}
\textbf{Esempi}: Due algoritmi sugli alberi binari che fanno uno delle BFS.
\begin{lstlisting}[caption= Esempi DFS alberi binari]
% conta i nodi di un albero binario
countNodi(TREE albero)
	if albero $==$ nil then
		return 0
	else 
		return 1 + countNodi(albero.left()) + countNodi(albero.right())

% stampa delle espressioni memorizzate in una struttura ad albero binario
stampa_exp(TREE albero)
	if albero.left() $==$ nil and albero.right() $==$ nil then
		print albero.read()
	else 
		print "("
		stampa_exp(albero.left())
		print albero.read()
		stampa_exp(albero.right())
		print ")"
\end{lstlisting}

\subsubsection{Esempi visite alberi binari con BFS}
Un albero binario di Natale è un albero binario tale per cui vale la seguente regola: tutti i livelli hanno la stessa brillantezza.  \\La brillantezza di un livello è pari alla somma della brillantezza dei nodi appartenenti a tale livello.  Un albero binario nil è un albero binario di Natale, ma triste.
\begin{lstlisting}[caption=Esempi BFS alberi binari]
boolean alberoBinNatale(TREE T)
	if T $==$ nil then
		return true
	else
		QUEUE q = new QUEUE
		q.enqueue(T)
		int brillantezzaAlbero = T.brillance
		int currentBrillance = 0
		int nodilivello = 1
		while not q.isEmpty() do
			TREE u = q.dequeue()
			currentBrillance = currentBrillance + u.brillance
			nodilivello = nodilivello - 1
			if u.left() $\neq$ nil then
				q.enqueue(u.left())
			if u.right() $\neq$ nil then
				q.enqueue(u.right())
			if nodilivello $==$ 0 then
				if currentBrillance $\neq$ brillantezzaAlbero then
					return false
				currentBrillance = 0
				nodilivello = q.size()
	return true
\end{lstlisting}
\newpage
\begin{flushleft}
La soluzione proposta conosce il numero di nodi che fanno parte di un livello, in quanto, nel momento in cui un livello termina, in coda sono presenti tutti i nodi del livello successivo. \\
Il primo livello ovviamente ha dimensione 1, in quanto è un albero binario radicato. Ovviamente partendo da questo è semplice sommare la brillantezza dei livelli sapendo che, al termine dello stesso, gli elementi del prossimo livello sono dati dagli n elmenti in coda a tal momento, dove $n = q.size()$, dove $q$ è l'instanza della coda.
\end{flushleft}
\subsection{Alberi generici}
Un albero generico è un albero con un numero di figli non fissato.\\
\textbf{Specifica}: Tutte le operazioni possibili sugli alberi generici sono le seguenti. 
\begin{lstlisting}[ caption= Specifica albero generico]
% Costruisce un nuovo nodo, contenente v, senza figli o genitori
Tree(ITEM v)
% Legge il valore memorizzato nel nodo
ITEM read()
% Modifica il valore memorizzato nel nodo
write(ITEM v)
% Restituisce il padre, oppure nil se questo nodo e' radice
TREE parent()
% Restituisce il primo figlio, oppure nil se questo nodo e' una foglia
TREE leftmostChild()
% Restituisce il prossimo fratello, oppure nil se assente
TREE rightSibling()
% Inserisce il sottoalbero t come primo nodo di questo nodo
insertChild(TREE t)
% Inserisce il sottoalbero t come prossimo fratello di questo nodo
insertSibling(TREE t)
% Distruggi l'albero radicato identificato dal primo figlio
deleteChild()
% Distruggi l'albero radicato identificato dal prossimo fratello
deleteSibling()
\end{lstlisting}
\textbf{Memorizzazione}: Esistono 3 diverse modalità per memorizzare un albero generico: vettore dei figli, primo figlio - prossimo fratello, vettore dei padri.\
\newpage

\subsubsection{Visite alberi generici}
\textbf{Problema}: Dato un albero generico T, visitare tutti i nodi della struttura.\
\textbf{Soluzione}: Esistono principalmente due metodologie per visitare un albero:
\begin{itemize}
	\item DFS
\end{itemize}
\begin{lstlisting}[ caption= Visita DFS albero generico]
preorder_generic_dfs(TREE albero)
	if albero $\neq$ nil then
		print albero.read()
		TREE tmp = albero.leftmostChild()
		while(tmp $\neq$ nil)
			preorder_generic_dfs(tmp)
			tmp = tmp.rightSibling()
    
postorder_generic_dfs(TREE albero)
	if albero $\neq$ nil then
		TREE tmp = albero.leftmostChild()
		while(tmp $\neq$ nil)
			postorder_generic_dfs(tmp)
			tmp = tmp.rightSibling()
		print albero.read()
\end{lstlisting}
\begin{itemize}
 	\item BFS
\end{itemize}
\begin{lstlisting}[ caption= Visita BFS albero generico]
generic_bfs(TREE albero)
	QUEUE q = new QUEUE
	q.enqueue(albero)
	while not q.empty() do
		TREE tmp = q.dequeue()
		print tmp.read()
		tmp = tmp.leftmostChild()
		while tmp $\neq$ nil do
			q.enqueue(tmp)
			tmp = tmp.rightSibling()
\end{lstlisting}
\newpage

\subsection{Alberi Binari di Ricerca (ABR)}
\textbf{Descrizione}: Un albero $T$ è un albero binario che rispetta le seguenti proprietà: le chiavi contenute nei nodi del \emph{sottoalbero sinistro} di un nodo generico $u$ sono minori di $u.key$, mentre le chiavi contenute nei nodi del \emph{sottoalbero destro} sono maggiori di $u.key$. L'idea degli ABR è quella di portare la ricerca binaria negli alberi. Un ABR è un dizionario in cui ogni nodo $u$ contiene delle associazioni chiave-valore salvate attraverso una coppia \emph{(u.key, u.value)}.\\
\textbf{Specifica}: La struttura di un ARB e tutte le operazioni possibili sono le seguenti. 
\begin{lstlisting}[ caption= Specifica ABR]
TREE
	TREE parent
	TREE left
	TREE right
	ITEM key
	ITEM value

TREE lookupNode(TREE T, ITEM k)
TREE insertNode(TREE T, ITEM k, ITEM v)
TREE removeNode(TREE T, Item k)
TREE successorNode(TREE t)
TREE predecessorNode(TREE t)
TREE min()
TREE max()
\end{lstlisting}
\textbf{Implementazione}: Implementazione di un ABR usando la specifica precedente. 
\begin{lstlisting}[ caption= Specifica ABR: lookup]
% lookup versione ricorsiva
TREE lookupNode(TREE T, ITEM k)
	if T $==$ nil or T.key $==$ k then
		return T
	else
		return lookupNode(iif(k < T.key, T.left, T.right), k)

% loopup versione iterativa
TREE lookupNode(TREE T, ITEM k)
	TREE u = T
	while u $\neq$ nil and u.key $\neq$ k do
		if k < u.key then
			u = u.left			% Sotto-albero sinistro
		else
			u = u.right			% Sotto-albero destro
	return u
\end{lstlisting}
\newpage
\begin{lstlisting}[ caption= Specifica ABR: min e max]
TREE min(TREE albero)
	TREE u = albero
	if u $==$ nil then 
		retun u
	while u.left $\neq$ nil do
		u = u.left
	return u

TREE max(TREE albero)
	TREE u = albero
	if u $==$ nil then 
		retun u
	while u.right $\neq$ nil do
		u = u.right
	return u
\end{lstlisting}
\begin{lstlisting}[ caption= Specifica ABR: successore e predecessore]
TREE successore(TREE albero)
	if albero $==$ nil then
		return albero
	if albero.right $\neq$ nil then
		return min(albero.right)
	else
    	% risalgo la catena di padri
		TREE parent = albero.parent
		while parent $\neq$ nil and parent.left $\neq$ albero do
			parent = parent.parent
			albero = albero.parent
	return parent
  
TREE predecessore(TREE albero)
	if albero $==$ nil then
		return albero
	if albero.left $\neq$ nil then
		return max(albero.left)
	else
		% risalgo la catena di padri
		TREE parent = albero.parent
		while parent $\neq$ nil and parent.right $\neq$ albero do
			parent = parent.parent
			albero = albero.parent
	return parent
\end{lstlisting}
\newpage
\begin{lstlisting}[ caption= Specifica ABR: insert e link]
TREE insert(TREE albero, ITEM key, ITEM value)
	if albero $==$ nil then
		albero = new TREE(key, value)
	% trovo la posizione del nodo da inserire
	TREE parent = albero
	TREE t = nil
	while t $\neq$ nil and t.key $\neq$ key do
	  	parent = t
		if key < t.key
    		t = t.left
		else 
			t = t.right
	% fermato avro' dunque padre e la posizione in cui inserire il figlio
	if t $\neq$ nil then
		% significa che devo solo aggiornare il valore
		t.value = value
	else
		% il nodo da inserire e' nuovo nell'albero, in questo caso va dunque inserito un nuovo nodo
		TREE nuovo = new TREE(k, value)
		link(parent, nuovo, k)
	return albero

% la procedura link e' importante solamente per l'inserimento del nuovo nodo, in quanto in base alla chiave stabilisce dove il nodo deve essere posizionato
link(TREE albero, TREE inserire, ITEM k)
	if k < albero.key then
		%  inserimento a sinistra
		albero.left = inserire
	else
		% significa che il nodo da inserire e' un nodo la cui chiave e' maggiore di quella del padre
		albero.right = inserire
	if inserire $\neq$ nil then
		inserire.parent = albero
\end{lstlisting}
\newpage
\begin{lstlisting}[ caption= Specifica ABR: cancellazione]
TREE cancellazione(TREE albero, ITEM key)
	% devo individuare il nodo da eliminare, mi basta quello
	TREE u = albero
	while u $\neq$ nil and u.key $\neq$ key do
		if key < u.key then
			u = u.left
		else
			u = u.right
	% controllo se il nodo da rimuovere ha veri e propri figli
	if u.left $==$ u.right $==$ nil then
		% caso albero formato da un unico nodo da rimuovere
		if u $==$ albero then
			return nil
		else
			link(u.parent, nil, u.key)
			delete u
	% caso invece un solo figlio
	if u.left $==$ nil then 
		if u $==$ albero then 
			u = u.left
			u.parent = nil
			return u
		else
			u = u.left
			u.parent = (u.parent).parent
	else if u.right $==$ nil then 
		if u $==$ albero then 
			u = u.right
			u.parent = nil
			return u
		else
			u = u.right
			u.parent = (u.parent).parent
	else 
		%  entrambi i figli
		TREE s = successore(u)
		link(s.parent, s.right, s.key)
		% sostituzione dei nodi successore e u
		u.key = s.key
		u.value = s.value
		delete s
\end{lstlisting}
\newpage

\subsection{Alberi Red-Black}
\textbf{Descrizione}: Sono dei particolari alberi binari di ricerca che hanno la caratteristica di essere bilanciati e di mantenere il bilanciamento tra l'altezza dele foglie anche in seguito ad inserimenti e cancellazioni. Le principali caratteristiche sono: ogni nodo è colorato di rosso o di nero, le chiavi vengono salvate solo nei nodi interni dell'albero e le foglie sono dei nodi speciali Nil. Per essere un albero red-black un ABR deve rispettare dei vincoli:
\begin{itemize}
 	\item La radice deve essere nera
 	\item Tutte le foglie sono nere
 	\item Entrambi i figli di un nodo rosso sono neri
 	\item Tutti i cammini semplici da un nodo u ad una delle foglie contenute nel sottoalbero radicato in u hanno lo stesso numero di nodi neri
\end{itemize}
\textbf{Implementazione}: Di seguito una parte di una possibile implementazione degli alberi red-black.
\begin{lstlisting}[ caption= Specifica Albero Red-Black]
TREE
	TREE parent
	TREE left
	TREE right
	int color
	ITEM key
	ITEM value

% le funzioni di rotazione vengono usate per mantenere le proprita' dell'albero RB	
TREE rotateLeft(TREE x)
	% prelevo right e left
	TREE y = x.right
	TREE p = x.parent
	% assegno B a figlio destro di x
	x.right = y.left 
	if y.left $\neq$ nil then
		y.left.parent = x
	% x diventa il figlio sinistro di y
	y.left = x
	x.parent = y
	% y diventa figlio di p
	y.parent = p
	if p $\neq$ nil then
		if p.left $==$ x then
			p.left = y
		else
			p.right = y
	return y
  	
% insert e balanceInsert 
	...
	
% cancellazione
	...
\end{lstlisting}

\end{document}