\documentclass[../cheatSheetAlgoritmi.tex]{subfiles}
\begin{document}

\section{Insiemi e dizionari}
\subsection{Insiemi con vettori booleani}
\textbf{Descrizione}: Un insieme è una collezione di m oggetti con valori compresi tra 1..m. Un insieme può essere memorizzato in un vettore booleano di m elementi, con l'indice che rappresenta l'elemento e il contenuto del vettore se l'elemento è presente o meno. I vantaggi sono: implementazione semplice e velocità nel verificare se un elemento appartiene all'insieme, mentre come svantaggi abbiamo che occupiamo O(m) memoria anche per salvare pochi elementi e che alcune operazioni vengono eseguite in O(m).\\
\textbf{Implementazione}: Possibile implementazione del Set attraverso vettori booleani.\
\begin{lstlisting}[ caption= Set con vettore booleano: operazioni base]
% definizione struttura dati SET
SET 
	boolean[] V
	int size	% numero di elementi presenti attualmente nel SET
	int dim		% numero di elementi che possono stare in totale nel SET
	
SET Set(int m)
	SET t = new SET
	t.size = 0
	t.dim = m
	t.V = [false] * m    % inizializzo tutti gli elementi a false
	return t
	
boolean contains(int x)
	if 1 $\leq$ x $\leq$ dim then
		return V[x]
	else
		return false
		
int size()
	return size

insert(int x)
	if 1 $\leq$ x $\leq$ dim then
		if not V[x] then
			size = size+1
			v[x] = true
			
remove(int x)
	if 1 $\leq$ x $\leq$ dim then 
		if V[x] then
			size = size-1
			v[x]= false
\end{lstlisting}
\newpage
\begin{lstlisting}[ caption= Set con vettore booleano: operazioni base]
% definizione struttura dati SET
SET union(SET A, SET B)
	SET C = Set(max(A.dim, B.dim))
	for i = 1 to A.dim do
		if A.contains(i) then
			C.insert(i)
	for i = 1 to B.dim do
		if B.contains(i) then
			C.insert(i)

SET intersection(SET A, SET B)
	SET C = Set(min(A.dim, B.dim))
	for i = 1 to min(A.dim, B.dim) do
		if A.contains(i) and B.contains(i) then
			C.insert(i)

SET difference(SET A, SET B)
	SET C = Set(A.dim)
	for i = 1 to A.dim do
		if A.contains(i) and not B.contains(i) then
			C.insert(i)	
\end{lstlisting}	
\textbf{Implementazioni reali}: In Java e C++ sono già implementati i set e sono già presenti nelle librerie standard con in nome BitSet. Esistono anche implementazioni dei set con liste, vettori non ordinati o strutture dati più complesse come alberi e hash table.

\subsection{Bloom Filters}
\textbf{Descrizione}: Struttura dati che unisce i vantaggi dei BitSet a quelli delle tabelle di Hash. Vantaggi: struttura dati dinamica e bassa occupazione di memoria. Svantaggi: non è possibile effettuare cancellazioni, le risposte ricevute dalla struttura sono probabilistiche e i dati non sono realmente memorizzati nella struttura.\\
\textbf{Specifica}: Sono presenti due principali operazioni:
\begin{itemize}
 	\item insert(k): inserisce l'elemento k nel bloom filter 
 	\item bool contains(k): restituisce false se l'elemento k è sicuramente  non presente nell'insieme, restituisce true se l'elemento può essere presente oppure no (falsi positivi). 
\end{itemize}
Per l'implementazione usiamo un vettore booleano A di m bit, inizializzato a false e k funzioni di hash: h$_{1}$, h$_{2}$, ..., h$_{k}$: U $\rightarrow$ [0, m-1]\
\begin{lstlisting}[ caption= Implementazione Bloom Filter]
insert(k)
	for i = 1 to k do
		A[hi(k)] = true
		
boolean contains(k)
	for i = 1 to k do
		if A[hi(k)] == false then 
			return false
	return true
\end{lstlisting}

\end{document}