\documentclass[../cheatSheetAlgoritmi.tex]{subfiles}
\begin{document}

\section{Algoritmi probabilistici}
\subsection{Introduzione}
Gli algoritmi probabilistici, prendono tale nome perché la casualità è un fattore determinante nella loro esecuzione. \\
Esistono algoritmi probabilistici di due diverse tipologie:
\begin{itemize}
	\item Algoritmi la cui correttezza è probabilistica (\textcolor{red}{Montecarlo})
	\item Algoritmi corretti, il cui tempo di funzionamento è probabilistico (\textcolor{red}{Las Vegas})
\end{itemize}
\subsection{Test di primalità}
Ci chiediamo quale potrebbe essere un algoritmo per determinare se un numero $n$ in input risulta primo oppure no. 
Un approccio basilare al problema potrebbe essere quello di considerare tutti i valori $i$ da 2 a $\lfloor \sqrt{n} \rfloor$, se uno di questi valori divide perfettamente $n$, ovvero il modulo della divisione $n/i$ da resto $0$, allora il numero $n$ non può essere primo. \\
Se non siamo in grado di trovare un divisore esatto di n durante questa analisi allora il numero considerato è sicuramente primo. 
 \begin{lstlisting}[ caption= Numero primo corretto ma inefficiente]
 boolean isPrimeNaif(int n)
 	for i = 2 to $\lfloor \sqrt{n} \rfloor$ do
 		if (n mod i) == 0 then
 			return false
 	return true
\end{lstlisting}
Approcciamo ora il problema in un modo differente, come alcuni di voi ricorderanno dal corso di \emph{Fondamenti matematici}, \emph{Il piccolo teorema di Fermat} consente di definire un interessante relazione per un numero primo $n$ qualsiasi: \\
\textbf{Il piccolo teorema di Fermat} \\
Se $n$ è un numero primo allora:
\begin{equation*}
    \forall b, 2 \leq b < n : \text{$b^{n-1}$ mod $n = 1$}
\end{equation*}
Sfruttiamo dunque tale reazione per controllare la primalità di un numero $n$ generico.
 \begin{lstlisting}[ caption= Numero primo non deterministico con 1 tentativo]
 boolean isPrime1(int n)
 	b = random(2, n - 1)
 	if ($b^{n-1}$ mod n) $\neq$ 1 then
 		return false
 	return true
\end{lstlisting}
Si noti che questo algoritmo non è corretto, in quanto la condizione dettata da tale enunciato va interpretata nel modo in cui è scritto. \\
In altre parole, se vale la conclusione per un singolo elemento non possiamo ricondurci al preambolo, esistono infatti dei numeri detti \textcolor{red}{pseudo-primi} che si comportano, rispetto ad alcune proprietà, in maniera analoga ai numeri primi tali per cui: 
\begin{align*}
    \exists b, 2 \leq b < n - 1: & \text{$b^{n-1}$ mod $n = 1$}
\end{align*}
Ad esempio:
\begin{align*} 
	 n = 341 = 11 \times 13 \\
    2^(340) \text{ mod } 341 = 1
\end{align*}
Possiamo quindi implementare l'algoritmo in modo tale che sia \emph{più corretto} ma comunque ancora non deterministico. Prendiamo dunque l'algoritmo precedente e lo estendiamo per $k$ valori, dove $k$ è un valore fissato. \\
Ancora una volta non si può avere la certezza della veridicità della soluzione, ma in quanto è provata per più valori \emph{potrebbe} avere maggiori occasioni per riconoscere un valore non primo.
 \begin{lstlisting}[ caption= Numero primo non deterministico con k tentativi]
 boolean isPrime2(int n, int k)
 	for i = 1 to k do
 		b = random(2, n - 1)
 		if ($b^{n-1}$ mod n) $\neq$ 1 then
 			return false
 	return true
\end{lstlisting}
Si ha infatti che se il risultato di tale esecuzione è \textbf{false} allora il numero $n$ è sicuramente composto, quindi non primo (non soddisfa la conclusione del piccolo teorema di Fermat), mentre invece se il valore di ritorno è \textbf{true} allora \textcolor{red}{potrebbe} essere primo. \\
\textbf{Il test di primalità di Miller-Rabin} \\
Un altro modo per testare la primalità è quello proposto dall'algoritmo di \emph{Miller-Rabin}, il quale è basato sul seguente teorema. \\
\textbf{Teorema}:
\begin{itemize}
	\item Se $n$ è primo, allora \textcolor{red}{per ogni} intero $b$, con $2 \leq b < n$ valgono \textcolor{red}{entrambe} le seguenti condizioni:
	\begin{itemize}
		\item $mcd(n, b) = 1$
		\item $b^m$ mod $n= 1 \lor \exists i, 0 \leq i < v : b^{m \cdot 2^i}$ mod $n = n - 1$
	\end{itemize}
	con $n - 1 = m \cdot 2^{v}$, $m$ dispari
\end{itemize}
Da cui è possibile ricavare la seguente contrapposizione. \\
\textbf{Contrapposizione}: \\
Se \textcolor{red}{esiste} un intero $b$, con $2 \leq b < n$ tale che \textcolor{red}{almeno una} delle seguenti condizioni è vera:
\begin{itemize}
	\item $mcd(n,b) \neq 1$
	\item $b^{m}$ mod  $n \neq 1 \land \forall i, 0 \leq i < v : b^{m \cdot 2^i}$ mod $n \neq n - 1$  
\end{itemize}
allora n è composto (quindi non primo). \\
Rabin ha dimostrato che se il numero $n$ è composto, allora ci sono almeno $3/4 (n - 1)$ testimoni in [2, ..., n - 1]. Un \emph{testimone} è un valore che attesta la non primalità di un numero, ossia che rende vera almeno una delle due affermazioni della contrapposizione. Il test per verificare se il valore è composto ha probabilità inferiore a $1/4$ di rispondere erroneamente. L'algoritmo sarà ancora non deterministico data la presenza di \emph{strong liar}, dei valori validano la contrapposizione nonostante il numero analizzato non sia primo. \\
L'implementazione di questo algoritmo è la seguente: 
 \begin{lstlisting}[ caption= Numero primo Miller-Rabin (Montecarlo)]
 boolean isPrimeMillerRabin(int n, int k)
 	int m, v = { si prende la rappresentazione di n - 1 in binario, esso sara' composto da un valore m (in binario) seguito da un numero v di zeri}
 	% Ad esempio per n - 1 = 228 = (110111 00) 
 	% si ha m = (110111)  = 57 e v = |00| = 2
 	for i = 1 to k do
 		b = random(2, n - 1)
 		if isComposite(n, b, m, v) then
 			return false
 	return true
 
 % Controlla se e' valida almeno una affermazione della contrapposizione
 boolean isComposite(int n, int b, int m, int v)
 	if mcd(n, b) $\neq$ 1 then
 		return true
 	if ($b^m$ mod n) $\neq$ 1 then
 		for i = 0 to v do
 			if ($b^{m \cdot 2^{i}}$ mod n) $==$ $n - 1$ then	
 				return false
 	return true
 	
 % Massimo comune divisore euclideo
 int mcd(a, b)
    if b $==$ 0
        return a
    else
        return mcd(b, a mod b)
\end{lstlisting}
Tale algoritmo ha una complessità computaizonale particolare che non discuteremo: \\
 $\mathcal{O}(k \log^2 n \log \log n \log \log \log n)$. \\
 Esso ha una probabilità di errore del $(1/4)^k$ dove $k$ è il numero di tentativi con cui viene ricercato il testimone. \\
 Nonostante non sia deterministico è comunque utilizzato dato il costo computazionale relativamente basso. \\
Come si può intuire tutti gli algoritmi proposti fino ad ora, fatta eccezione per la prima soluzione deterministica, sono algoritmi della categoria \emph{Montecarlo} in quanto la loro correttezza è probabilistica, ovvero siamo sempre sicuri di non essere in presenza di numeri primi se viene ritornato \textbf{false}, non possiamo dire di esserlo se viene restituito \textbf{true}.
\subsection{Algoritmi probabilistici: statistica}
I casi analizzati sono i più semplici, cioè quelli che ci permettono di trovare alcune caratteristiche che sono statisticamente rilevanti da un vettore numerico. \\
Alcuni esempi sono:
\begin{itemize}
	\item Media: $\mu = \frac{1}{n} \sum_{i = 1}^{n} A[i]$
	\item Varianza: $\sigma^2 = \frac{1}{n} \sum_{i = 1}^{n} (A[i] - \mu)^2$
	\item Moda: il valore (o i valori) più frequenti
\end{itemize} 
\subsection{Il calcolo della mediana}
Il problema della mediana è quello di calcolare, dato un vettore $A$ contenene $n$ valori, l'elemento/i che occuperebbe/ro la posizione centrale se il vettore $A$ fosse ordinato. \\
Ovviamente è possibile riordinare il vettore mediante un algoritmo di ordinamento basato su confronti di costo computazionale pari a $\mathcal{O}(n \log n)$ e prelevare l'elemento a metà. \\
Ci chiediamo ora se è possibile fare meglio di così, estendendo il problema del calcolo della mediana a quello della selezione, dato che la mediana è un suo sottoproblema: \\
Dato un array $A$ contenente $n$ valori e un valore $1 \leq k \leq n$, trovare l'elemento che occuperebbe la posizione $k$ se il vettore fosse ordinato. \\
Per valori di $k$ piccoli è possibile utilizzare uno \emph{heap binario}, costruendolo ed eliminando i valori fino all'elemento $k$.
 \begin{lstlisting}[ caption= Selezione per piccoli valori di k]
 int heapSelect(ITEM[] A, int n, int k)
	buildHeap(A)
	for i = 1 to k - 1 do
		deleteMin(A, n)
	return deleteMin(A, n) 	
\end{lstlisting}
La complessità è pari a $\mathcal{O}(n + k \log n)$, dove se $k = \mathcal{O}(n / \log n)$, il costo è $\mathcal{O}(n)$ mentre se $k \geq n/2$ abbiamo ottenuto un algoritmo peggiore rispetto a quello che effettua l'ordinamento degli elementi. \\
Proviamo ad ideare un algoritmo basato su un approccio \emph{divide et impera}, molto simile a quanto fatto per il \emph{Quicksort}. \\
Dato che è molto simile ad un problema di ricerca possiamo intuire che non è necessario cercare il valore k-esimo in entrambe le metà in cui il vettore è stato diviso, ma solamente in una di esse stando però attenti agli indici.
 \begin{lstlisting}[ caption= Algoritmo di selezione]
 ITEM selection(ITEM[] A, int start, int end, int k) 	
 	if start $==$ end then
 		return A[start]
 	else 
 		int j = pivot(A, start, end)
 		int q = j - start + 1
 		if k $==$ q then
 			return A[j]
 		else if k < q then
 			return selection(A, start, j - 1, k)
 		else
 			return selection(A, j + 1, end, k - q)
\end{lstlisting}
Questo algoritmo infatti, sfrutta il calcolo del \emph{pivot}, visto quando abbiamo introdotto il quickSort, e in base a tale valore decide che parte del vettore andare ad analizzare. \\
La complessità di questo algoritmo di selezione è rappresentata qui sotto divisa per i tre casi di studio: \\
\textbf{Caso pessimo}: 
\begin{equation*}
    T(n)=\begin{cases}
        1 & \text{$n \leq 1$}\\
        T(n - 1) + n & \text{$n > 1$}
    \end{cases}
\end{equation*}
Il cui costo computazionale è $\mathcal{O}(n^2)$ \\
\textbf{Caso ottimo}: 
\begin{equation*}
    T(n)=\begin{cases}
        1 & \text{$n \leq 1$}\\
        T(n/2) + n & \text{$n > 1$}
    \end{cases}
\end{equation*}
Il cui costo computazionale è $\mathcal{O}(n)$. \\
La complessità computazionale nel \textbf{caso medio} invece, sotto l'assunzione che la procedura \emph{pivot} restituisca con la stessa probabilità una qualunque posizione $j$ del vettore A:
\begin{align*}
	 T(n) &= n + \frac{1}{n} \sum_{q = 1}^{n} T(max\{q - 1, n - q\}) &\text{Media su n casi} \\
	& \leq n + \frac{1}{n} \sum_{q = \lfloor n/2 \rfloor}^{n-1} 2T(q)  &\text{per $n > 1$} 
\end{align*}
Procediamo dunque al calcolo della complessità:
\begin{align*}
	 T(n) & \leq n + \frac{1}{n} \sum_{q = \lfloor n/2 \rfloor}^{n-1} 2  \cdot cq \leq n + \frac{2c}{n} \sum_{q = \lfloor n/2 \rfloor}^{n-1} q & \text{Sostituzione, raccolgo il fattore 2} \\
	& \leq n + \frac{2c}{n} \sum_{q = n/2 - 1}^{n-1} q & \text{Estensione della sommatoria} \\
	& = n + \frac{2c}{n} \left( \sum_{q = 1}^{n-1} q -  \sum_{q = 1}^{n/2 - 2} q \right) & \text{Sottrazione della prima parte} \\
	& = n + \frac{2c}{n} \left( \frac{n (n - 1}{2} - \frac{(n/2 - 1)(n/2 - 2)}{2}\right) &\\
	& = n + \frac{c}{n} \left(n^2 - n - (1/4 n^2 - 3/2n + 2) \right) & \\
	& = n + \frac{c}{n} \left(3/4n^2 + 1/2 n - 2 \right) & \\
	& \leq n + \frac{c}{n} \left(3/4n^2 + 1/2 n \right) & \\
	& = n + 3cn/4 +1/2 \leq cn & \text{Vera per $c \geq 6$ e $n \leq 1$} \\
\end{align*}
Siamo partiti dall’assunzione che $j$ assume equiprobabilisticamente tutti i valori compresi fra $1$ e $n$, se così non fosse possiamo forzarlo noi scambiando qualche valore: \\
$A[random(start,end)] \leftrightarrow A[start]$ (vale anche per QuickSort)\\
La complessità nel caso medio è dunque: $\mathcal{O}(n)$ nel caso della Selezione, mentre $O(n \log n)$ nel caso dell’Ordinamento. \\
Questo algoritmo, basato sul calcolo del $pivot$ è un algoritmo di tipo \emph{Las Vegas} in quanto corretto, ma il tempo di esecuzione è probabilistico, a seconda della distribuzione di valori ritornati dalla funzione $pivot$. \\
\textbf{Selezione deterministica}: \\
L'idea è la seguente:
\begin{itemize}
	\item Suddividiamo il vettore in gruppi di 5.
	\item Chiameremo l’i-esimogruppo $S_i$, con $i \in [1, \lceil n/5 \rceil]$. 
	\item Troviamo il mediano $M_i$ di ogni gruppo $S_i$. 
	\item Tramite una chiamata ricorsiva, si trova il mediano $m$ delle mediane $[M_1,M_2,...,M_{\lceil n/5 \rceil}]$. 
	\item Usiamo $m$ come pivot e richiamiamo l'algoritmo ricorsivamente sulla pozione di vettore opportuna, come nella \emph{selection()}.
	\item Quando la grandezza del vettore analizzato scende sotto una certa dimensione (nel nostro caso inferiore o uguale a 10 elementi), possiamo utilizzare un algoritmo di ordinamento per trovare il mediano (si usa insertionSort in quanto adatto per vettori di dimensione molto ridotta).
\end{itemize}
\begin{lstlisting}[caption=Mediana deterministica in $\mathcal{O}(n)$]
ITEM select(ITEM[] A, int start, int end, int k)
	% Se la dimensione risulta essere inferiore ad una soglia (10) 
	% ordina il vettore e restituisci il k - esimo elemento di A[start..end]
	if $end - start + 1 \leq 10$ then 
		insertionSort(A, start, end)
		return A[start + k - 1]
	% Divide A in n/5 sottovettori di dimensione 5 e ne calcola la mediana
	ITEM[] M = new ITEM[1..$\lceil n/5 \rceil$]
	for i = 1 to $\lceil n/5 \rceil$ do
		M[i] = median5(S, start + i (i - 1) $\cdot$ 5, end)
	% Individua la mediana delle mediane e usala come perno
	ITEM m = select(M,1,$\lceil n/5 \rceil$, $\lceil \lceil n / 5 \rceil / 2 \rceil$)
	int j = pivot(S, start, end, m)
	% Calcola l'indice q di m in[start...end]
	% Confronta q con l'indice cercato e ritorna il valore conseguente
	int q = j - start + 1
	if q $==$ k then
		return m
	else if q < k then
		return select(A, start, q - 1, k)
	else
		return select(A, q + 1, end, k - q)

ITEM median5(ITEM[] A, int start, int end)
	ITEM median
	ITEM tmp 
	% Precomputing
	% swap S[start] with S[start + 1] if S[start] < S[start + 1] 
	if S[start] > S[start + 1] then
		tmp = S[start]
		S[start] = S[start +1]
		S[start + 1] = tmp
	% swap S[start + 3] with S[start + 4] if S[start + 3] < S[start + 4] 
	if S[start + 3] > S[start + 4] then
		tmp = S[start + 3]
		S[start + 3] = S[start + 4]
		S[start + 4] = tmp
	% swap S[start + 3] with S[start] if S[start + 3] < S[start] 
	if S[start + 3] < S[start] then
		tmp = S[start]
		S[start] = S[start + 3]
	 	S[start + 3] = tmp
	 	% swap S[start + 1] with S[start + 4]
		tmp = S[start + 1]
		S[start + 1] = S[start + 4]
		S[start + 4] = tmp 
	% Median finding
	if S[start + 2] > S[start + 1] then
		if S[start + 1] < S[start + 3] then
			median = min(S[start + 2], S[start + 3]) 
		else 
			median = min(S[start + 1], S[start + 4])
	else
		if S[start + 2] > S[start + 3] 
			median = min(S[start + 2], S[start + 4]) 
		else 
			median = min(S[start + 1], S[start + 3]) 
	return median
\end{lstlisting}
Considerazioni:
\begin{itemize}
	\item Il calcolo dei mediani $M[ ]$ richiede al più $6 \lceil n/5 \rceil$ confronti. 
	\item La prima chiamata ricorsiva dell'algoritmo \emph{select()} viene effettuata su  $\lceil n/5 \rceil$ elementi.
	\item La seconda chiamata ricorsiva dell’algoritmo \emph{select()} viene effettuata al massimo su $7n/10$ elementi (esattamente $n - 3 \lceil \lceil n/5 \rceil /2 \rceil$).  
\end{itemize}
L'algoritmo \emph{select()} esegue nel caso pessimo $\mathcal{O}(n)$ confronti:
\begin{equation*}
    T(n) = T(n/5) + T(7n/10) + 11/5 n
\end{equation*}
Quest'ultimo algoritmo è deterministico, dunque esegue in tempo lineare su qualsiasi input, tuttavia molto spesso gli viene preferito quello \emph{Las Vegas} in quanto possiede fattori moltiplicativi più bassi e ha comunque un tempo \emph{atteso} $\mathcal{O}(n)$, con caso peggiore $\mathcal{O}(n^2)$.
\end{document}