\documentclass[../cheatSheetAlgoritmi.tex]{subfiles}
\begin{document}

\chapter{Lo Pseudocodice}
Per descrivere gli algoritmi, ovvero delle procedure composte da operazioni elementari e che devono risolvere un problema computazionale in un tempo finito, dobbiamo ricorrere ad un linguaggio pressoché formale, in quanto il linguaggio naturale può generare delle ambiguità.
\subsection{Notazione in pseudocodice}
\begin{itemize}
  	\item $a=b$: assegnamento del valore di b alla variabile a;
  	\item $a \leftrightarrow b$: scambio di valori all'interno delle variabili;
  	\item $T[] A = new T[1..n]$: creazione di un array del tipo dato T;
  	\item $T[][] A = new$ $T[1..n][1..n]$: creazione di una matrice di tipo dato T;
  	\item \textbf{int, float, boolean, char}: tipi di dato che utilizzeremo nella nostra pseudocodifica;
  	\item $==, \neq, \geq, \leq, >, <$: operatori di confronto;
  	\item $+, -, \cdot, /, \lfloor x \rfloor, \lceil x \rceil, \log, x^2$: classici operatori matematici;
  	\item \textit{iif(condizione, $v_1$, $v_2$)}: operatore ternario;
  	\item \textbf{if} \textit{condizione} \textbf{then} istruzione: operatore di selezione;
  	\item \textbf{if} \textit{condizione} \textbf{then} $istruzione_1$ \textbf{else} $istruzione_2$
  	\item \textbf{while} \textit{condizione} \textbf{do} istruzione: ciclo while;
  	\item \textbf{foreach} $\textit{elemento} \in \textit{insieme}$ \textbf{do} istruzione
  	\item \textbf{for} $i = 1$ \textbf{to} $n$ \textbf{do} istruzione: ciclo for;
  	\item \textbf{for} $i = n$ \textbf{downto} $1$ \textbf{do} istruzione: ciclo for al contrario;
  	\item \textbf{return}
  	\item \textit{\% commento}
  	\item $CLASS$ \textit{variabile} = new $CLASS$: istanziazione di una classe;
  	\item $\textit{r.campo}= 10$: accesso ad un campo di una struttura;
  	\item \textbf{delete} variabile: per la cancellazione;
	\item $r = \textbf{nil}$: assegnazione ad una variabile di tipo puntatore il valore di nil
\end{itemize}

\end{document}