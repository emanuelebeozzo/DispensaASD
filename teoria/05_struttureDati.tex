\documentclass[../cheatSheetAlgoritmi.tex]{subfiles}
\begin{document}

\section{Strutture dati base}
Nelle seguenti sottosezioni saranno elencate le specifiche (dati e operazioni) delle strutture dati base che saranno usate nei capitoli successivi.

\subsection{Insiemi}
\begin{lstlisting}[caption= Specifica SET]
% Restituisce la cardinalita' dell'insieme
int = size()
% Restituisce true se x e' contenuto nell'insieme
boolean contains(ITEM x)
% Inserisce x nell'insieme, se non gia' presente
insert(ITEM x)
% Rimuove x dall'insieme, se presente
remove(ITEM x)
% Restituisce un nuovo insieme che e' l'unione di A e B
SET union(SET A, SET B)
% Restituisce un nuovo insieme che e' l'intersezione di A e B
SET intersection(SET A, SET B)
% Restituisce un nuovo insieme che e' la differenza di A e B
SET difference(SET A, SET B)
\end{lstlisting}

\subsection{Dizionari}
\begin{lstlisting}[ caption= Specifica dizionari]
% Restituisce il valore associato alla chiave k se presente, nil altrimenti
ITEM lookup(ITEM k)
% Associa il valore v alla chiave k
insert(ITEM k, ITEM v)
% Rimuove l'associazione della chiave k
remove(ITEM k)
\end{lstlisting}
L'implementazione più conosciuta è l'HashTable. 

\subsection{Stack}
\begin{lstlisting}[ caption= Specifica STACK]
% Restituisce true se la pila e' vuota
boolean isEmpty()
% Inserisce v in cima alla pila
push(ITEM v)
% Estrae l'elemento in cima alla pila e lo restituisce al chiamante
ITEM pop()
% Legge l'elemento in cima alla pila
ITEM top()
\end{lstlisting}

\newpage

\subsection{Queue}
\begin{lstlisting}[ caption= Specifica QUEUE]
% Restituisce true se la coda e' vuota
boolean isEmpty()
% Inserisce v in fondo alla coda
enqueue(ITEM v)
% Estrae l'elemento in testa alla coda e lo restituisce al chiamante
ITEM dequeue()
% Legge l'elemento in testa alla coda
ITEM top()
\end{lstlisting}

\end{document}