\documentclass[../cheatSheetAlgoritmi.tex]{subfiles}
\begin{document}

\section{Equazioni di Ricorrenza}
\subsection{Metodo dell'Albero di Ricorsione, o per Livelli}
Srotoliamo la ricorrenza in un albero i cui nodi rappresentano i costi ai vari livelli di ricorsione.\\
\textbf{Primo Esempio}
\begin{center}
	\begin{equation*}
  		T(n)=\begin{cases}
    		4T(n/2) + n & \text{$n > 1$}\\
    		1 & \text{$n \leq 1$}
  		\end{cases}
	\end{equation*}
\end{center}

T(n) = n + 4T(n/2) = n + 4(4T(n/4) + n) = n + 4n/2 + 16T($n/2^{2}$) = n + 2n + 16n/4 + 64T(n/8)\\
= ...\\
= n + 2n + 4n + ... + $2^{k-1}$ + $4^{k}T(n/2^{k})$ \textcolor{red}{se poniamo $2^{k} = n \implies k = \log{n}$}\\
= n + 2n + 4n + ... + $2^{\log{n-1}}n$ + $4^{\log{n}}T(1)$ = n $\sum\limits_{j=0}^{\log{n-1}} {2^{j}} + 4^{\log{n}}$\\
= n \textcolor{red}{$\frac{2^{\log{n}-1}}{2-1}$} + $ 4^{\log{n}}$ = $n(n-1) + n^{2} = 2n^{2} - n$ = \textcolor{red}{$\Theta(n^{2})$}\\\\
\textbf{Secondo Esempio}
\begin{center}
	\begin{equation*}
  		T(n)=\begin{cases}
    		4T(n/2) + n^{3} & \text{$n > 1$}\\
    		1 & \text{$n \leq 1$}
  		\end{cases}
	\end{equation*}
\end{center}

Proviamo a visualizzare l'albero delle chiamate:

\begin{center}
	\renewcommand{\arraystretch}{1.2}
	\begin{tabular}{ |c|c|c|c|c| } 
		\hline
			Livello & Dim. & Costo chiam. & N. chiamate & Costo livello \\ 
		\hline
			0 & n & $n^3$ & 1 & $n^3$ \\ 
		\hline
			1 & $n/2$ & $(n/2)^{3}$ & 4 & $4(n/2)^{3}$\\ 
		\hline
			2 & $(n/4)$ & $(n/4)^{3}$ & 16 & $16(n/4)^{3}$\\
		\hline
			... & ... & ... & ... & ...\\
		\hline
			i & $n/2^{i}$ & $(n/2^{i})^{3}$ & $4^{i}$ & $4^{i}(n/2^{i})^{3}$\\
		\hline
			... & .	.. & ... & ... & ...\\
		\hline
			$l-1$ & $n/2^{l-1}$ & $(n/2^{l-1})^{3}$ & $4^{l-1}$ & $4^{l-1}(n/2^{l-1})^{3}$\\
		\hline
			$l = \log{n}$ & $1$ & $T(1)$ & $4^{\log{n}}$ & $4^{\log{n}}$\\
		\hline
	\end{tabular}
\end{center}
Dunque sappiamo che al livello $\log{n}$ il costo sarà pari a $4^{\log{n}}$, duqnue ci basta fare la sommatoria fino a $\log{n-1}$ e poi aggiungere il restante:\\
T(n) =  $\sum\limits_{i=0}^{\log{n-1}} {4^{i} \cdot n^{3}/2^{3i}} + 4^{\log{n}}$\\\\
\textcolor{red}{possiamo portare fuori i termini costanti dalla sommatoria}\\\\
= \textcolor{red}{$n^{3}$} $\sum\limits_{i=0}^{\log{n-1}} {\frac{4^{i}}{2^{3i}}} + 4^{\log{n}}$ = $n^{3}$ $\sum\limits_{i=0}^{\log{n-1}} {\frac{\textcolor{red}{2^{2i}}}{2^{3i}}} + \textcolor{red}{2^{2\log{n}}}$ = $n^{3}$ $\sum\limits_{i=0}^{\log{n-1}} {\textcolor{red}{(\frac{1}{2})^{i}}} + \textcolor{red}{n^{2}}$\\\\\\
\textcolor{red}{estendiamo la sommatoria fino a $\infty$}\\\\
\textcolor{red}{$\leq$} $n^{3}$ $\sum\limits_{i=0}^{\textcolor{red}{\infty}} {(\frac{1}{2})^{i}} + n^{2}$ = 
$n^{3}$ $\cdot$ \textcolor{red}{$\frac{1}{1-\frac{1}{2}}$} + $n^{2}$ = $2n^{3} + n^2$\\\\
Abbiamo dimostrato che T(n) = $\mathcal{O}$($n^{3}$): non possiamo affermare che sia $\Theta(n) = n^{3}$ perchè abbiamo dimostrato solo un limite superiore.\\
Tuttavia basta notare che l'equazione originale possiede un termine del tipo $an^{3}$ (con a $\in$ $\mathbb{N}$) $\implies$ T(n) = $\Omega(n^{3})$\\  
Visto che abbiamo dimostrato anche il limite inferiore $\implies$ T(n) = $\Theta(n^{3})$ 
\subsection{Metodo di Sostituzione, o per Tentativi}
Metodo in cui si cerca di indovinare una soluzione, in base alla propria esperienza, e si dimostra la correttezza di questa soluzione tramite \textcolor{red}{induzione}\\\\
\textbf{Primo Esempio}
\begin{center}
	\begin{equation*}
  		T(n)=\begin{cases}
    		T(\lfloor n/2 \rfloor) + n & \text{$n > 1$}\\
    		1 & \text{$n \leq 1$}
  		\end{cases}
	\end{equation*}
\end{center}
\hfill\\
\textbf{Limite Superiore}\\
\textcolor{red}{Tentativo: T(n) = $\mathcal{O}$(n)}\\
$\exists c$ $>$ 0, $\exists m$ $\geq$ 0 : T(n) $\leq$ cn, $\forall n$ $\geq$ m\\\\
\textbf{Caso Base}: Dimostriamo la disequazione per T(1)\\
T(1) = 1 $\stackrel{?}{\leq}$ 1 $\cdot$ c $\iff$ c $\geq$ 1\\\\
\textbf{Ipotesi Induttiva}: $\forall k$ $<$ n : T(k) $\leq$ ck\\
\textbf{Passo di Induzione}: Dimostriamo la disequazione per T(n)\\
T(n) =  T($\lfloor n/2 \rfloor$) + n $\leq$ $c( \lfloor n/2 \rfloor)$ + n $\leq$ cn/2 + n $\stackrel{?}{\leq}$ cn $\iff$  c/2 + 1 $\leq$ c\\
$\iff$ c $\geq$ 2\\
Abbiamo provato che T(n) $\leq$ cn e un valore di c che rispetta entrambe le disequazioni è ad esempio \textcolor{red}{c = 2}.\\
Questo vale per \textcolor{red}{n = 1}, e per tutti i valori di n seguenti; quindi \textcolor{red}{m = 1}.\\
Abbiamo provato che T(n) = $\mathcal{O}$(n), ora dobbiamo occuparci del \textbf{Limite Inferiore}\\\\
\textcolor{red}{Tentativo: T(n) = $\Omega$(n)}\\
$\exists c$ $>$ 0, $\exists m$ $\geq$ 0 : T(n) $\geq$ cn, $\forall n$ $\geq$ m\\\\
\textbf{Caso Base}: Dimostriamo la disequazione per T(1)\\
T(1) = 1 $\stackrel{?}{\geq}$ 1 $\cdot$ c $\iff$ c $\leq$ 1\\\\
\textbf{Ipotesi Induttiva}: $\forall k$ $<$ n : T(k) $\geq$ ck\\
\textbf{Passo di Induzione}: Dimostriamo la disequazione per T(n)\\
T(n) =  T($\lfloor n/2 \rfloor$) + n $\geq$ $c( \lfloor n/2 \rfloor)$ + n $\geq$ cn/2 \textcolor{red}{- 1} + n\\\\\\
\textcolor{red}{sottraggo 1 in modo che sia sicuramente più piccolo del limite inferiore} \\\\
cn/2 - 1 + n $\stackrel{?}{\geq}$ cn $\iff$ c/2 - 1/n + 1 $\geq$ c $\iff$ c $\leq$ 2 - 2/n\\\\
Mettendo assieme le due condizioni si ha che nel caso base (per n $\leq$ 1) c $\leq$ 1 e nel passo induttivo (per n $>$ 1) c $\leq$ 2 - 2/n (quindi nel caso peggiore, cioè per n = 2, si ha c $\leq$ 1): un valore c che soddisfa entrambe le disequazioni per un n $\geq$ 1 è c = 1. Questo vale ovviamente per n = 1 e quindi per m = 1.\\
Notare che era possibile dimostrare che T(n) = $\Omega(n)$ come fatto nel caso del Metodo per Livelli, cioè notando che nell'equazione fosse presente un termine del tipo  $an^{1}$ (con a $\in$ $\mathbb{N}$).\\
Visto che abbiamo dimostrato sia il Limite Inferiore che il Limite Superiore possiamo dire che T(n) = $\Theta(n)$.\\\\
\textbf{Secondo Esempio (Difficoltà Matematica)}
\begin{center}
	\begin{equation*}
  		T(n)=\begin{cases}
    		T(\lfloor n/2 \rfloor) + T(\lceil n/2 \rceil) + 1 & \text{$n > 1$}\\
    		1 & \text{$n \leq 1$}
  		\end{cases}
	\end{equation*}
\end{center}
\hfill\\
\textcolor{red}{Tentativo: T(n) = $\mathcal{O}$(n)}\\
$\exists c$ $>$ 0, $\exists m$ $\geq$ 0 : T(n) $\leq$ cn, $\forall n$ $\geq$ m\\\\
\textbf{Ipotesi Induttiva}: $\forall k$ $<$ n : T(k) $\leq$ ck\\
\textbf{Passo di Induzione}: Dimostriamo la disequazione per T(n)\\
T(n) =  T($\lfloor n/2 \rfloor$) + T($\lceil n/2 \rceil$) + 1 $\leq$ c($\lfloor n/2 \rfloor$) + c($\lceil n/2 \rceil$) + 1\\
$\leq$ 2c(n/2) + 1 $\stackrel{?}{\leq}$ cn $\iff$ \cancel{cn} + 1 $\leq$ \cancel{cn} $\iff$ 1 $\leq$ 0 \textcolor{red}{NO!}\\\\
Pur non sembrando, il tentativo è corretto, tuttavia non riusciamo a dimostrarlo per via di un termine di ordine inferiore, usiamo un ipotesi induttiva più stretta:\\
\textcolor{red}{$\exists b$ $>$ 0, $\forall k$ $<$ n : T(k) $\leq$ ck - b}\\
T(n) =  T($\lfloor n/2 \rfloor$)+ T($\lceil n/2 \rceil$) + 1 $\leq$ c($\lfloor n/2 \rfloor$) - b + c($\lceil n/2 \rceil$) - b + 1\\
$\leq$ 2c(n/2) - 2b + 1 = cn - 2b + 1 $\stackrel{?}{\leq}$ cn - b $\iff$ b $\geq$ 1\\\\
\textbf{Caso Base}: Dimostriamo la disequazione per T(1)\\
T(1) = 1 $\stackrel{?}{\leq}$ 1 $\cdot$ c - b $\iff$ c $\geq$ b + 1\\\\
$\implies$ una coppia di valori che può soddisfare le disequazioni è b = 1 e c = 2; questo vale da n = 1 in poi e quindi per m = 1.\\\\
\textbf{Terzo Esempio (Problemi con i Casi Base)}
\begin{center}
	\begin{equation*}
  		T(n)=\begin{cases}
    		2T(\lfloor n/2 \rfloor) + n & \text{$n > 1$}\\
    		1 & \text{$n \leq 1$}
  		\end{cases}
	\end{equation*}
\end{center}
\hfill\\
\textcolor{red}{Tentativo: T(n) = $\mathcal{O}(n\log{n})$}\\
$\exists c$ $>$ 0, $\exists m$ $\geq$ 0 : T(n) $\leq$ c($n\log{n}$), $\forall n$ $\geq$ m\\\\
\textbf{Ipotesi Induttiva}: $\forall k$ $<$ n : T(k) $\leq$ c($k\log{k}$)\\
\textbf{Passo di Induzione}: Dimostriamo la disequazione per T(n)\\
T(n) = 2T($\lfloor n/2 \rfloor$) + n $\leq$ 2($c \lfloor n/2 \rfloor \log{\lfloor n/2 \rfloor}$) + n\\
$\leq$ 2($c(n/2) \log{(n/2)}$) + n = 2$cn$($\log{n}$ - $\log{2}$) + $n$\\
= $cn$($\log{n}$ - 1) + $n$ $\stackrel{?}{\leq}$ $cn\log{n}$ $\iff$ $c(\log{n}-1) + 1 \leq c \log{n}$\\
$\iff$ c $\geq$ 1\\\\
\textbf{Caso Base}: Dimostriamo la disequazione per T(1)\\
T(1) = 1 $\stackrel{?}{\leq}$ $c$ $\cdot$ $\log{1}$ $\iff$ 1 $\leq$ 0 \textcolor{red}{NO!}\\
...ma, non è un problema: il valore di m lo possiamo scegliere noi.\\
Proviamo gli altri casi:\\
T(2) = 2T($\lfloor 2/2 \rfloor$) + 2 = 2T(1) + 2 = 4 $\stackrel{?}{\leq}$ $c$ $\cdot$ $2 \log{2}$ $\iff$ c $\geq$ 2\\
T(3) = 2T($\lfloor 3/2 \rfloor$) + 3 = 2T(1) + 3 = 5 $\stackrel{?}{\leq}$ $c$ $\cdot$ $3 \log{3}$ $\iff$ c $\geq$ $\frac{5}{3 \log{3}}$\\
T(4) = 2T($\lfloor 4/2 \rfloor$) + 4 = 2T(2) + 4\\
\textcolor{red}{...ma a questo punto mi posso fermare perchè ho trovato un termine noto}\\
A questo punto non resta che scegliere il valore maggiore per cui vale c, in questo caso $c$ $\geq$ 2\\
Quindi abbiamo dimostrato che vale per n = 2 in poi e una coppia è n = 2, c = 2 (e quindi m = 2).\\

\subsection{Metodo dell'Esperto, o delle Ricorrenze Comuni}
Siano $a$ e $b$ due costanti intere tali che $a$ $\geq$ 1 e $b$ $\geq$ 2 e, $c, \beta$ costanti reali tali che $c$ $>$ 0 e $\beta$ $\geq$ 0. Sia T(n) data dalla relazione di ricorrenza:
\begin{center}
	\begin{equation*}
  		T(n)=\begin{cases}
    		aT(n/b) + cn^{\beta} & \text{$n > 1$}\\
    		d & \text{$n \leq 1$}
  		\end{cases}	
	\end{equation*}
\end{center}

Posto $\alpha$ = $\log{a}/\log{b}$ = $\log_{b}{a}$, allora:\\

\begin{center}
	\begin{equation*}
  		T(n)=\begin{cases}
			\Theta(n^{\alpha}) & \text{$\alpha > \beta$}\\
    		\Theta(n^{\alpha} \log{n}) & \text{$\alpha = \beta$}\\
    		\Theta(n^{\beta}) & \text{$\alpha < \beta$}
  		\end{cases}
	\end{equation*}
\end{center}

\textbf{Esempio}:\\
$T(n) = 4T(n/16) + n^{2}$ $\implies$ $a = 4$, $b = 16$\\
$\beta = 2$\\
$\alpha = \log{b} / \log{a} = \log{16} / \log{4} = \log_{16}{4} = 1/2$\\
Visto che $\beta > \alpha \implies T(n) = \Theta(n^{\beta}) = \Theta(n^{2})$ 
\newpage

\subsection{Ricorrenze Lineari con Partizione Bilanciata (Estesa)}
Sia $a$ $\geq$ 1, $b$ $>$ 1, $f(n)$ asintoticamente positiva, e sia 
\begin{center}
	\begin{equation*}
  		T(n)=\begin{cases}
    		aT(n/b) + f(n) & \text{$n > 1$}\\
    		d & \text{$n \leq 1$}
  		\end{cases}
	\end{equation*}
\end{center}

Siano dati 3 casi:

\begin{center}
	\renewcommand{\arraystretch}{1.2}
	\begin{tabular}{ |c|c| } 
		\hline
			(1) & $\exists \epsilon$ $>$ 0 : $f(n) = \mathcal{O}(n^{\log_{b}{a} - \epsilon}) \implies T(n) = \Theta(n^{\log_{b}{a}})$\\
		\hline
			(2) & $f(n) = \Theta(n^{\log_{b}{a}}) \implies T(n) = \Theta(f(n) \log{n})$\\
		\hline
			(3) & \makecell{$\exists \epsilon$ $>$ 0 : $f(n) = \Omega(n^{log_{b}{a} + \epsilon})$ $\wedge$ \\ $\exists c$ : 0 $<$ $c$ $<$ 1, $\exists m$ $>$ 0 : \\ $af(n/b) \leq cf(n), \forall n$ $\geq$ $m$} $\implies T(n) = \Theta(f(n))$ \\
		\hline
	\end{tabular}
\end{center}

\textbf{Esempio 1}\\
$T(n) = 9T(n/3) + n \implies a = 9, b = 3, \log_{b}{a} = 2$\\
$f(n) = n = \mathcal{O}(n^{\log_{b}{a} - \epsilon}) =  \mathcal{O}(n^{2 - \epsilon})$, con $\epsilon$ $<$ 1\\
Caso (1) $\implies$ $T(n) = \Theta(n^{2})$\\\\
\textbf{Esempio 2}\\
$T(n) = T(2n/3) + 1 \implies a = 1, b = 2/3, \log_{b}{a} = 0$\\
$f(n) = n^{0} = \Theta(n^{\log_{b}{a}}) =  \Theta(n^{0})$\\
Caso (2) $\implies$ $T(n) = \Theta(\log{n})$\\\\
$\textbf{Esempio 3}$\\
$T(n) = 3T(n/4) + n\log{n} \implies a = 3, b = 4, \log_{b}{a} \approx 0.79$\\
$f(n) = n\log{n} = \Omega(n^{\log_{4}{3} + \epsilon})$, con $\epsilon$ $<$ 1 - $\log_{4}{3} \approx 0.208$\\\\
Caso(3) $\implies$ Dobbiamo dimostrare che:\\
$\exists c$ $\leq$ 1, $\exists m$ $>$ 0 : $af(n/b) \leq cf(n), \forall n$ $\geq$ $m$\\
$af(n/b) = 3n/4\log{n/4} = 3n/4\log{n} - 3n/4\log{4} \leq 3n/4\log{n} \stackrel{?}{\leq} cn\log{n}$\\
L'ultima disequazione è soddisfatta da $c = 3/4$ e $m = 1$

\subsection{Ricorrenze Lineari di Ordine Costante}
Siano $a\textsubscript{$1$}$, $a\textsubscript{$2$}$, $...$, $a\textsubscript{$h$}$ costanti intere non negative, con $h$ costante positiva, $c$ e $\beta$ costanti reali tali che $c$ $>$ 0 e $\beta$ $\geq$ 0, e sia $T(n)$ definita dalla relazione di ricorrenza:
\begin{center}
	\begin{equation*}
  		T(n)=\begin{cases}
     		\sum\limits_{1 \leq i \leq h} {a \textsubscript{$i$} T(n-i)} + cn^{\beta} & \text{$n > m$}\\
    		\Theta(1) & \text{$n \leq m \leq h$}
  		\end{cases}
	\end{equation*}
\end{center}

Posto $a = \sum\limits_{1 \leq i \leq h} {a \textsubscript{$i$}}$, allora: 

\begin{itemize}
	\item $T(n)$ è $\Theta(n^{\beta+1})$, se $a = 1$,
	\item $T(n)$ è $\Theta(a^{n}n^{\beta})$, se $a \geq 2$,
\end{itemize}
\textbf{Esempio 1}\\
$T(n) = T(n - 10) + n^{2} \implies a = 1, \beta = 2$\\
Caso (1) poichè $a = 1$, il costo è polinomiale $\implies$ $T(n) = \Theta(n^{\beta+1}) = \Theta(n^{3})$\\\\
\textbf{Esempio 2}\\
$T(n) = T(n - 2) +  T(n - 1) + 1 \implies a = 2, \beta = 0$\\
Caso (2) poichè $a = 2$, il costo è esponenziale $\implies$ $T(n) = \Theta(a^{n}n^{\beta}) = \Theta(2^n)$

\end{document}