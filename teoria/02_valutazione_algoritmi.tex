\documentclass[../cheatSheetAlgoritmi.tex]{subfiles}
\begin{document}

\section{Sottovettore di Somma Massima: Algoritmo di Kadane}
\textbf{Problema}: Dato un vettore di interi $A[1...n]$, restituire il sottovettore $A[i...j]$ (con $i, j < n$ e $i \leq j$) di somma massimale, ovvero la cui somma degli elementi $\sum_{k=i}^j A[k]$ è più grande o uguale alla somma degli elementi di qualunque altro sottovettore.\\
\textbf{Soluzione} (con Programmazione Dinamica): \textbf{Algoritmo di Kadane}
\begin{lstlisting}[caption=Kadane Algorithm]
int maxSumKadane(int[] A, int n)
	int maxSoFar = 0
	int maxHere = 0
	for i = 1 to n do
		maxHere = max(maxHere + A[i], 0)
		maxSoFar = max(maxSoFar, maxHere)
	return maxSoFar
\end{lstlisting}

\section{Valutazione Algoritmi} \subsection{Calcolo del Minimo}
\textbf{Problema}: Dato un vettore di interi $S[1...n]$ trovare il minimo\\
\textbf{Soluzione}:
\begin{lstlisting}[ caption=Ricerca Minimo]
int min(int[] S, int n)
	int min = S[1]
	for i=2 to n do
		if S[i] < min then
			min = S[i]
	return min
\end{lstlisting}
Nella versione seguente viene ritornata una coppia $<$$int$, $int$$>$ che contiene il minimo e la sua posizione:
\begin{lstlisting}[caption=Ricerca Minimo ritornando gli indici]
(int, int) min(int[] S, int n)
	int min = S[1]
	int posMin = 1;
	for i=2 to n do
		if S[i] < min then
			min = S[i]
			minPos = i
	return (min, minPos)
\end{lstlisting}
\newpage

\subsection{Ricerca Binaria}
\textbf{Problema}: Dato un vettore ordinato di interi $S[1...n]$ e un intero $v$, controllare se nel vettore è presente un elemento $A[i] = v$ (con $0 \leq i \leq n$) e in quel caso restituirne l'indice.\\
\textbf{Soluzione} (Ricerca Binaria): Analizzo l'elemento centrale del sottovettore (indice $m$)
\begin{itemize}
	\item $S[m]=v$ allora ho trovato il valore cercato;
	\item $S[m]<v$ allora vuol dire che dovrò cercare il valore nella metà di destra;
	\item $S[m]>v$ allora vuol dire che dovrò cercare il valore nella metà di sinistra
\end{itemize}
\begin{lstlisting}[ caption=Ricerca Binaria]
int binarySearch(int[] S, int v, int start, int end)
	if start > end then
		return 0
	else
		int m = $\lfloor$(start + end)/2$\rfloor$
		if S[m] $==$ v then
			return m
		else if S[m] < v then
			return binarySearch(S, v, m+1, end)
		else
			return binarySearch(S, v, start, m-1)
\end{lstlisting}

\section{Analisi degli Algoritmi}
\subsection{Tempo di Calcolo}
\begin{itemize}
	\item Ogni istruzione richiede un tempo costante per essere eseguita;
	\item Questa costante è potenzialmente diversa per ogni istruzione;
	\item Ogni istruzione viene eseguita un certo numero (\#) di volte che dipende da codice a codice
\end{itemize}
\textbf{Funzione min()}
\begin{lstlisting}[caption=Ricerca Minimo + calcolo del costo]
int min(int[] S, int n)			% Costo # Volte
	int min = S[1]	    		% c1   	 1
	for i = 2 to n do		% c2     n
		if S[i] < min then 	% c3 	 n-1
			min = S[i]	% c4 	 n-1
	return min			% c5	 1
\end{lstlisting}
$T(n) = c1 + c2n + c3(n-1) + c4(n-1) + c5 = (c2 + c3 + c4)n + (c1 + c5 - c3 - c4) = \textcolor{red}{an + b}$

\newpage

\subsection{Complessità Algoritmi vs Complessità Problemi: Moltiplicazione di Gauss}
\textbf{Problema: Moltiplicazione numeri complessi}
\begin{itemize}
	\item $(a + bi)(c + di) = [ac - bd] + [ad + bc]i$
	\item Input: $a, b, c, d$
	\item Output: $ac - bd, ad + bc$
\end{itemize}
Se consideriamo un modello di calcolo dove la moltiplicazione costa 1 mentre le addizioni e le sottrazioni costano 0.01 otteniamo che il costo dell'algoritmo risulta essere 2*(2 + 0.01) = 4.02\\
Usiamo la tecnica della \textbf{moltiplicazione di Gauss}:
\begin{itemize}
	\item $A_1 = a \cdot c$
	\item $A_2 = b \cdot d$
	\item $m = (a + b) \cdot (c + d)$
	\item $A_3 = m - A_1 - A_2 = (a + b) \cdot (c + d) - (a \cdot c) - (b \cdot d) = \cancel{ac} + ad + bc + \cancel{bd} - \cancel{ac} -\cancel{bd} = ad + bc$
	\item $A_4 = A_1 - A_2 = ac - bd$
\end{itemize}
Notiamo che $A_3$ e $A_4$ sono esattamente i termini che ci aspettiamo vengano restituiti dalla moltiplicazione dei numeri complessi e che, in questo caso, il costo delle operazioni all'interno del nostro sistema risulta essere 3.05.

\subsection{Notazione Asintotica e Proprietà}
Definzione \textcolor{red}{$\mathcal{O}(g(n))$}:\\
Sia $g(n)$ una funzione di costo; indichiamo con $\mathcal{O}(g(n))$ un insieme di funzioni $f(n)$ tali per cui:
\begin{center}
$\exists c > 0, \exists m \geq 0$ : $\textcolor{red}{f(n) \leq cg(n)} , \forall n \geq m$
\end{center}
Definizione \textcolor{red}{$\Omega(g(n))$}:\\
Sia $g(n)$ una funzione di costo; indichiamo con $\Omega(g(n))$ un insieme di funzioni $f(n)$ tali per cui:
\begin{center}
$\exists c > 0, \exists m \geq 0$ : $\textcolor{red}{f(n) \geq cg(n)} , \forall n \geq m$
\end{center}
Definizione \textcolor{red}{$\Theta(g(n))$}:\\
Sia $g(n)$ una funzione di costo; indichiamo con $\Omega(g(n))$ un insieme di funzioni $f(n)$ tali per cui:
\begin{center}
$\exists c_1 > 0, \exists c_2 > 0, \exists m \geq 0$ : $\textcolor{red}{c_1g(n) \leq f(n) \leq c_2g(n)} , \forall n \geq m$
\end{center}
\newpage
\begin{flushleft}
\textbf{Regola Generale:}
\end{flushleft}
\begin{itemize}
	\item \textbf{Espressioni Polinomiali}\\
		$\textcolor{red}{a_{k} n^k + a_{k-1} n^{k-1} + ... + a_{1}n + a_{0}, a_{k} > 0 \implies f(n) = \Theta(n^k)}$
		\begin{itemize}
			\item \textbf{Limite Superiore}: $\exists c > 0, \exists m \geq 0$ : $f(n) \leq cn^k, \forall n \geq m$
			
			$f(n) = a_{k} n^k + a_{k-1} n^{k-1} + ... + a_{1}n + a_{0}$
			
			$\leq$ $a_{k}n^k$ $+$ $\mid$$a_{k-1}$$\mid$$n^{k-1}$ $+$ ... $+$ $\mid$$a_{1}$$\mid$$n$ $+$ $\mid$$a_{0}$$\mid$ 
			
			$\leq$ $a_{k}n^k$ $+$ $\mid$$a_{k-1}$$\mid$$n^{k}$ $+$ ... $+$ $\mid$$a_{1}$$\mid$$n^k$ $+$ $\mid$$a_{0}$$\mid$$n^k$ 
			
			$\leq$ $(a_{k}$ $+$ $\mid$$a_{k-1}$$\mid$ $+$ ... $+$ $\mid$$a_{1}$$\mid$ $+$ $\mid$$a_{0}$$\mid$$)n^k$
			
			$\leq cn^k$
			
			Che risulta essere vera per $c$ $\geq$ $(a_{k}$ $+$ $\mid$$a_{k-1}$$\mid$ $+$ ... $+$ $\mid$$a_{1}$$\mid$ $+$ $\mid$$a_{0}$$\mid$$)$ e per $m = 1$.
			
			\item \textbf{Limite Inferiore}: $\exists d > 0, \exists m \geq 0$ : $f(n) \geq cn^k, \forall n \geq m$
			
			$f(n) = a_{k} n^k + a_{k-1} n^{k-1} + ... + a_{1}n + a_{0}$
			
			$\geq$ $a_{k}n^k$ $-$ $\mid$$a_{k-1}$$\mid$$n^{k-1}$ $-$ ... $-$ $\mid$$a_{1}$$\mid$$n$ $-$ $\mid$$a_{0}$$\mid$ 
			
			$\geq$ $a_{k}n^k$ $-$ $\mid$$a_{k-1}$$\mid$$n^{k-1}$ $-$ ... $-$ $\mid$$a_{1}$$\mid$$n^{k-1}$ $-$ $\mid$$a_{0}$$\mid$$n^{k-1}$ 
			
			$\geq dn^k$
			
			L'ultima equazione è vera se:
			
			$d$ $\leq$ $a_{k}$ $-$ $\mid$$a_{k-1}$$\mid$$n^{-1}$ $-$ ... $-$ $\mid$$a_{1}$$\mid$$n^{-1}$ $-$ $\mid$$a_{0}$$\mid$$n^{-1} > 0 \iff n >$ $(\mid$$a_{k-1}$$\mid$ $+$ ... $+$ $\mid$$a_{0}$$\mid)/a_{k}$
		\end{itemize}
	\item \textbf{Casi Particolari}
	
		La complessità di \textcolor{red}{$f(n) = 5$} e di \textcolor{red}{$f(n) = 5 + \sin(n)$} è \textcolor{red}{ $\Theta$(1)}
		
		(P.S. la stessa cosa vale se vi è una limitazione al numero di dati che è necessario fornire in input)
	\item ... altri casi trattati nella parte di teoria
\end{itemize}

\subsection{Complessità}
$\forall r < s, h < k, a < b$\\
\textcolor{red}{$\mathcal{O}(1)$ $\subset$ $\mathcal{O}$($\log$$^{r}n$) $\subset$ $\mathcal{O}$($\log$$^{s}n$) $\subset$ $\mathcal{O}$($n^{h}$) $\subset$ $\mathcal{O}$($n^{h}$$\log$$^{r}n$)  $\subset$ $\mathcal{O}$($n^{h}$$\log$$^{s}n$) $\subset$ $\mathcal{O}$($n^{k}$) $\subset$ $\mathcal{O}$($a^{n}$) $\subset$ $\mathcal{O}$($b^{n}$)}
\newpage
\end{document}
