\documentclass[../cheatSheetAlgoritmi.tex]{subfiles}
\begin{document}

\section{Ricerca locale}
\textbf{Definizione}\\
Se si conosce una soluzione ammissibile ad un problema di ottimizzazione, si può provare a trovare una soluzione migliore nelle "vicinanze" di quella precendente. Si continua in questo modo fino a quando non si è più in grado di migliorare. \\
Uno schema possibile per la ricerca locale può essere il seguente:
\begin{lstlisting}[ caption= Ricerca locale]
ricercaLocale()
	$Sol$ = una soluzione ammissibile del problema
	while $\exists S$ $\in$ I($Sol$) migliore di $Sol$ do
		$Sol = S$
	return $Sol$
\end{lstlisting}
\subsection{Il problema del flusso massimo}
\textbf{Rete di flusso}:\\
Una \textcolor{red}{rete di flusso $G = (V, E, s, t, c)$} è data da:
\begin{itemize}
	\item un grafo orientato $G = (V, E)$
	\item un nodo $s \in V$ detto \textcolor{red}{sorgente}
	\item un nodo $t \in V$ detto \textcolor{red}{pozzo}
	\item una funzione di \textcolor{red}{capacità} $c : V \times V \rightarrow \mathbb{R}^{\geq 0}$, tale per cui $(u, v) \notin E \rightarrow c(u, v) = 0$
\end{itemize}
\textbf{Assunzioni}
\begin{itemize}
	\item Per ogni nodo $v \in V$, esiste un cammino $s \rightsquigarrow v \rightsquigarrow t$ da $s$ a $t$ che passa per $v$;
	\item Possiamo ignorare i nodi che non godono di questa proprietà.
\end{itemize}
\textbf{Flusso}: \\
Un \textcolor{red}{flusso} in $G$ è una funzione \textcolor{red}{$f: V \times V \rightarrow \mathbb{R}$} che soddisfa le seguenti proprietà:
\begin{itemize}
	\item \textcolor{red}{Vincolo sulla capacità}: $\forall u, v \in V : f(u, v) \leq c(u,v)$.\\ Il flusso non deve eccedere la capacità sull'arco.
	\item \textcolor{red}{Assimmetria}: $\forall u, v \in V, f(u,v) = - f(v, u)$. 
	\item \textcolor{red}{Conservazione del flusso}: $\forall u \in V - \{s, t\}, \sum_{v \in V} f(u, v) = 0$ 	\\
	Per ogni nodo, la somma dei flussi entranti deve essere uguale alla somma dei flussi uscenti.
\end{itemize}
\textbf{Valore del flusso}:\\
Il \textcolor{red}{valore di un flusso} $f$ è definito come: 
\begin{equation*}
  	|f| = \sum_{(s, v) \in E} f(s, v)
\end{equation*}
ovvero come la quantità di flusso uscente da $s$.
\newpage
\begin{flushleft}
\textbf{Flusso massimo}:\\
Data una rete G = $(V, E, s, t, c)$, trovare un flusso che abbia valore massimo fra tutti i flussi associabili alla rete.
\begin{equation*}
	|f^{*}| = max\{|f|\}
\end{equation*}
\end{flushleft}
\textbf{Flusso nullo}:\\
Definiamo \textcolor{red}{flusso nullo} la funzione $f_{0} : V \times V \rightarrow \mathbb{R}^{\geq 0}$ tale che $f(u, v) = 0$ per ogni $u, v \in V$ . \\\\
\textbf{Somma di flussi}:\\
Per ogni coppia di flussi $f_1$ e $f_2$ in $G$, definiamo il \textcolor{red}{flusso somma} $g = f_1 + f_2$ come un flusso tale per cui $g(u, v) = f_1(u, v) + f_2(u, v)$. \\\\
\textbf{Capacità residua}:\\
Definamo \textcolor{red}{capacità residua} di un flusso $f$ in una rete $G = (V, E, s, t, c)$ una funzione $c_f : V \times V \rightarrow \mathbb{R}^{\geq 0}$ tale che $c_f(u, v) = c(u, v) - f(u, v)$. \\
Si noti che per la definzione di capactià residua si creano degli \textcolor{red}{archi all'indietro}. \\\\
\textbf{Reti residue}:\\
Data una rete di flusso $G = (V, E, s, t, c)$ e un flusso $f$ su $G$, possiamo costruire una \textcolor{red}{rete residua} $G_f = (V, E_f, s, t, c_f)$, tale per cui $(u, v) \in E_f$ se e solo se $c_f(u, v) > 0$. \\\\
\textbf{Algoritmo, schema generale}:
 \begin{lstlisting}[ caption= schema generale flusso massimo]
int[][] maxFlow(GRAPH G, NODE s, NODE t, int[][] c)
	$f = f_0$
	$r = c$
	repeat 
		$g =$ trova un flusso in $r$ tale che $|g| > 0$, altrimenti $f_0$
		$f = f + g$
		$r =$ Rete di flusso residua del flusso $f$ in $G$
	until $g = f_0$ 
	return $f$
\end{lstlisting}
\textbf{Dimostrazione correttezza} \\\\
\textbf{Lemma}: Se $f$ è un flusso in $G$ e $g$ è un flusso in $G_f$, allora $f+g$ è un flusso in $G$. \\\\
\textbf{Dimostrazione}:
\begin{itemize}
	\item Vincolo sulla capacità:
		\begin{align*}
		g(u, v) & \leq c_f(u, v) \\
		\textcolor{red}{f(u, v)} + g(u, v) & \leq c_f(u, v) + \textcolor{red}{f(u, v)} \\
		(f + g)(u, v) & \leq c(u, v) - f(u, v) + f(u, v) \\
		(f + g)(u, v) & \leq c(u, v)
		\end{align*}
	\item Antisimmetria:
		\begin{align*}
		f(u, v) + g(u, v) & = -f(v, u) - g(v, u) \\
		f(u, v) + g(u, v) & = -(f(v, u) + g(v, u)) \\
		(f+g)(u, v) & = - (f + g)(v, u)
		\end{align*}
	\item Conservazione:
		\begin{align*}
		\sum_{v \in V}(f + g)(u, v) & = 	\sum_{v \in V}(f(u, v) + g(u, v)) \\
		& = \sum_{v \in V}f(u, v) +  \sum_{v \in V}g(u, v)\\
		& = 0
		\end{align*}
\end{itemize}
\textbf{Algoritmo di Ford-Fulkerson} \\
Tale algoritmo suggerisce che per trovare un flusso aumentante è necessario trovare un cammino $C = v_0, v_1, ..., v_n$ con $s = v_0$ e $t = v_n$ nella rete residua $G_f$.
Si identifica la capacità del cammino, corrispondente alla minore capacità degli archi incontrati (collo di bottiglia):
\begin{equation*}
  	c_f(C) = min_{i = 2...n} c_f(v_{i-1}, v_i)
\end{equation*}
A partire da questo si crea dunque un flusso addizionale $g$ tale che:
\begin{itemize}
	\item $g(v_{i-1}, v_i) = c_f(C)$;
	\item $g(v_{i-1}, v_i) = - c_f(C)$ (per antisimmetria)
	\item $g(x, y) = 0$  per tutte le altre coppie $(x, y)$
\end{itemize}
 \begin{lstlisting}[ caption= schema generale flusso massimo]
int[][] maxFlow(GRAPH G, NODE s, NODE t, int[][] c)
	int[][] f = new int[][]	% flusso parziale
	int[][] g = new int[][] % flusso da cammino aumentante
	int[][] r = new int[][] % rete residua
	foreach $u, v \in$ G.V() do
		f[u][v] = 0
		r[u][v] = c[u][v]
	repeat
		g = flusso associato ad un cammino aumentante in $r$, oppure $f_0$
		foreach $u, v \in G.V()$ do
			f[u][v] = f[u][v] + g[u][v] 	% f = f + g
			r[u][v] = c[u][v] - f[u][v] 	% Calcola $c_f$
	until $g = f_0$
	return f
\end{lstlisting}
Ford e Fulkerson suggerirono una semplice visita del grafo, in profondità oppure in ampiezza. \\Edmonds e Karp suggerirono di utilizzare una visita in ampiezza. \\
Il costo della visita: $\mathcal{O}(V + E)$ \\\\
\textbf{Complessità, limite superiore - Versione Ford-Fulkerson} \\
Se le capacità sono intere, l'algoritmo di Ford-Fulkerson ha complessità $\mathcal{O}((V + E)|f^{*}|)$ (liste) o $\mathcal{O}(V^2 |f^*|)$ (matrice).
\begin{itemize}
	\item L'algoritmo parte dal flusso nullo e termina quando il valore totale del flusso raggiunge $|f^*|$
	\item Ogni incremento del flusso aumenta il flusso di almeno un'unità
	\item Ogni ricerca di un cammino richiede una vistita del gafo, con costo $\mathcal{O}(V + E)$ o $\mathcal{O}(V^2)$
	\item La somma dei flussi e il calcolo della rete residua può essere effettuato in tempo $\mathcal{O}(V + E) $ o $\mathcal{O}(V^2)$.
\end{itemize}
\textbf{Complessità, limite superiore - Versione Edmonds e Karp} \\
Se le capacità sono intere, l'algoritmo di Edmonds e Karp ha complessità $\mathcal{O}(VE^2)$ nel caso pessimo. 
\begin{itemize}
	\item Vengono eseguiti $\mathcal{O}(VE)$ aumenti di flusso, ognuno dei quali richiede una visita in ampiezza $\mathcal{O}(V + E)$.
	\item $\mathcal{O}(VE(V + E)) = \mathcal{O}(VE^2)$
\end{itemize}
\textbf{Che complessità scegliere?}\\
Qualora impiegassimo una visita in ampiezza (BFS) per individuare il cammino aumentante nella rete residua facciamo uso sia dell'algoritmo di Ford-Fulkerson che di quello di Edmonds e Karp. \\
In questo caso dunque si presentano due limiti superiori per lo stesso algoritmo, da una parte abbiamo $\mathcal{O}((V + E)|f^{*}|)$ (liste di adiacenza) o $\mathcal{O}(V^2 |f^*|)$ (matrice di adiacenza), mentre dall'altra  $\mathcal{O}(VE^2)$.\\
Entrambi sono limiti superiori validi, dunque per determinare la complessità dell'algoritmo è sufficiente scegliere il più basso. \\
Dunque, utilizzando una visita in ampiezza la complessità è data da $\mathcal{O}(min((V + E)|f^{*}|, VE^2))$ (liste di adiacenza),\\ $\mathcal{O}(min(V^2|f^{*}|, VE^2))$ (matrice di adiacenza).
\subsection{Teoremi e lemmi utili}
In questa sottosezione sono mostrati alcuni Lemmi e Teoremi utili per la dimostrazione della complessità e della correttezza dell'algoritmo di Edmonds e Karp, che non verranno dimostrate per brevità.\\\\
\textbf{Lemma - Monotonia} \\
Sia $\delta f(s,v)$ la distanza minima da $s$ a $v$ in una rete residua $G_f$. \\Sia $f'=f+g$ un flusso nella rete iniziale, con $g$ flusso non nullo derivante da un cammino aumentante. Allora $\delta f'(s,v)\geq \delta f(s,v)$. \\\\
\textbf{Lemma - Aumenti di flusso} \\
Il numero totale di aumenti di flusso eseguiti dall’algoritmo di Edmonds e Karp è $\mathcal{O}(V E)$ \\\\
\textbf{Taglio} \\
Un \textcolor{red}{taglio} $(S,T)$ della rete di flusso $G= (V, E, s, t, c)$ è una partizione di $V$ in due sottoinsiemi disgiunti $S, T$ tali che: 
\begin{align*}
  	S = V - T \\
  	s \in S \land t \in T
\end{align*}
\textbf{Capacità di un taglio}: \\
La \textcolor{red}{capacità} $C(S, T)$ attraverso il taglio $(S,T)$ è pari a:
\begin{equation*}
  	C(S, T) = \sum_{u \in S, v \in T} c(u, v)
\end{equation*}
\textbf{Flusso di un taglio}: \\
Se $f$ è un flusso in $G$, il flusso netto $F(S,T)$ attraverso $(S,T)$ è: 
\begin{equation*}
  	F(S,T) = \sum_{u \in S, v \in T}f(u,v)
\end{equation*}
\textbf{Lemma - Valore del flusso di taglio} \\
Dato un flusso $f$ e un taglio $(S,T)$, la quantità di flusso $F(S,T)$ che attraversa il taglio è uguale a $|f|$. \\\\
\textbf{Lemma – Capacità taglio} \\
Il \textcolor{red}{flusso massimo} è limitato superiormente dalla capacità del \textcolor{red}{taglio minimo}, ovvero il taglio la cui capacità è minore fra tutti i tagli. \\\\
\textbf{Teorema conclusivo}
\begin{itemize}
	\item $f$ è un \textcolor{red}{flusso massimo};
	\item non esiste nessun cammino aumentante per $G$;
	\item esiste un \textcolor{red}{taglio minimo} $(S, T)$ tale che $C(S, T) = |f|$.
\end{itemize}
\subsection{Esempio di flusso massimo}
\textbf{Problema - Job Assignment} \\
\textbf{Input}: 
\begin{itemize}
	\item Un insieme $J$ contenente $n$ job;
	\item Un insieme $W$ contenente $m$ worker;
	\item Una relazione $R \subseteq J \times W$, tale per cui $(j, w) \in \mathbb{R}$ se e solo se il job $j$ può essere eseguito dal worker $w$.
\end{itemize}
\textbf{Output}:\\
 Il più grande sottoinsieme $O \subseteq \mathbb{R}$, tale per cui:
\begin{itemize}
	\item ogni job venga assegnato al più ad un worker;
	\item ad ogni worker venga assegnato al più un job.
\end{itemize}
\textbf{Soluzione:} \\
Si può intuire che il problema richiami le reti di flusso dato che esso fa riferimento ad un problema di matching, in particolare di trovare il più grande sottoinsieme che soddisfa tali abbinamenti. \\
Per questi tipi di problemi è sufficiente definire un accurata descrizione del processo risolutivo, senza dover scrivere l'algoritmo per la risoluzione. \\
Per risolvere il job assignment :\\ 
Si costruisce una rete di flusso $G = (V, E, s, p, c)$:
\begin{itemize}
	\item $V$ è l'insieme di nodi rappresentato da job e worker;
	\item $s$ è un nodo aggiuntivo, detto supersorgente;
	\item $p$ è un nodo anch'esso aggiuntivo detto superpozzo;
	\item $E$ è l'insieme di archi definiti come segue:
		\begin{itemize}
			\item Si aggiunge un arco tra job e worker se esso può essere eseguito dal corrispondente lavoratore;
			\item Si aggiunge un arco dalla supersorgente a tutti i job;
			\item Si aggiunge un arco da tutti i worker al superpozzo.
		\end{itemize}
	\item c (funzione di capacità) è definita come segue:
		\begin{itemize}
			\item c(s, u) = 1 per la sorgente a tutti i nodi job, dato che ogni job deve essere eseguito una singola volta;
			\item c(u, v) = 1 per ogni arco tra job e worker in quanto il lavoro deve essere assegnato al più un worker;
			\item c(v, p) = 1 per ogni arco tra worker e superpozzo in quanto ad ogni worker può essere assegnato al più un job.
		\end{itemize}
\end{itemize}
Il flusso massimo della rete di flusso, definita precedentemente, darà come risultato il massimo sottoinsieme cercato. \\
Complessità del problema:\\ il massimo flusso ottenibile in questa rete di flusso è dell'ordine di $\mathcal{O}(V)$, rappresentato dal numero di job presenti. Supponendo che il grafo G sia memorizzato con liste di adiacenza l'algoritmo di Ford-Fulkerson esegue in un tempo pari a $\mathcal{O}((V + E)V) = \mathcal{O}(VE)$ (nota bene: la complessità considerata è la seguente, e non quella definita da Edmonds e Karp in quanto superiore).
\end{document}
