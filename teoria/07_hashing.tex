\documentclass[../cheatSheetAlgoritmi.tex]{subfiles}
\begin{document}

\chapter{Hashing}
\textbf{Descrizione}: L'hash è la struttura dati più efficiente che implementa il concetto di dizionario: consente di memorizzare insiemi dinamici di coppie $<$chiave, valore$>$. Le coppie vengono indicizzate in base alla chiave. La struttura si basa su una funzione di hash che mappa ogni chiave k dell'insieme universo in un intero h(k)
e un array  che verrà usato per salvare i dati. Quando due o più chiavi hanno lo stesso valore hash, diciamo che è avvenuta una collisione. Esistono due principali metodi di gestione delle collisioni: 
\begin{itemize}
 	\item Liste di trabocco: le chiavi con lo stesso valore vengono memorizzate in una lista monodirezionale accessibile a partire dall'array (puntatore alla testa della lista). È fondamentale bilanciare la lunghezza delle liste di trabocco altrimenti si perde tutta l'efficienza della struttura. 
 	\item Indirizzamento aperto: tutte le chiavi vengono memorizzate nel vettore stesso, senza strutture aggiuntive. Se lo slot scelto per l'inserimento p già utilizzato, se ne cerca uno "alternativo". Esistono diverse tecniche di ricerca con indirizzamento aperto: ispezione lineare (agglomerazione primaria), ispezione quadratica e doppio hashing. 
\end{itemize}

\subsection{Doppio hashing}
\textbf{Descrizione}: Usa una funzione del tipo: H(k,i) = (H$_{1}$(k) + i $\cdot$ H$_{2}$(k)) mod m, con m lunghezza del vettore e H$_{1}$, H$_{2}$ funzioni di hash.H$_{1}$ fornisce la prima ispezione, H$_{2}$ le ispezioni successive. Un doppio hash permette le seguenti operazioni: inserimento, cancellazione (usando uno speciale valore deleted) e ricerca. 
\begin{lstlisting}[ caption= Hashing Doppio: struttura]
% Definizione struttura
HASH
	ITEM[] K		% Tabella delle chiavi
	ITEM[] V		% Tabella dei valori
	int m			% Dimensione della tabella
\end{lstlisting}
\newpage
\begin{lstlisting}[ caption= Hashing Doppio: specifica]
HASH Hash(int dim)
	HASH t = new HASH
	t.m = dim
	t.K = new ITEM[0...dim-1]
	t.V = new ITEM[0...dim-1]
	for i = 0 to dim-1 do
		t.K[i] = nil
	return t
	
int scan(ITEM k, boolean insert)
	int c = m		% Prima posizione deleted
	int i = 0		% Numero di ispezione
	int j = H1(k)	% Posizione attuale
	while K[j] $\neq$ k and K[j] $\neq$ nil and i < m do
		if K[j] == deleted and c == m then
			c = j
		j = (j + H2(k)) mod m
		i = i + 1
	if insert and K[j] $\neq$ k and c < m then
		j = c
	return j	
	
ITEM lookup(ITEM k)
	int i = scan(k, false)
	if K[i] == k then
		return V [i]
	else
		return nil
		
insert(ITEM k, ITEM v)
	int i = scan(k, true)
	if K[i] == nil or K[i] == deleted or K[i] == k then
		K[i] = k
		V [i] = v
	else
		% Errore: tabella hash piena
		
remove(ITEM k)
	int i = scan(k, false)
	if K[i] == k then
		K[i] = deleted
\end{lstlisting}

\newpage
\end{document}