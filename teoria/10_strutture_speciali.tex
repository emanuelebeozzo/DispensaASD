\documentclass[../cheatSheetAlgoritmi.tex]{subfiles}
\begin{document}

\section{Strutture dati speciali}
\subsection{Heap}
\textbf{Descrizione}: Struttura dati che associa i vantaggi dell'albero a quelli di un array. Gli alberi binari max-heap (rispettivamente min-heap) sono alberi binari completi tali per cui il valore memorizzato in ogni nodo è maggiore (rispettivamente minore) dei valori memorizzati nei suoi figli. Un albero binario si dice completo se:
\begin{itemize}
 	\item Tutte le foglie hanno profondità h o h-1 
 	\item Tutti i nodi a livello h sono accatastati a sinistra (non può mancare il figlio sinistro ed esserci il destro)
 	\item Tutti i nodi interni (non foglia) hanno grado 2, eccetto al più uno
 	\item Se il numero di nodi è n, l'altezza è h = $\lfloor$log \textit{n}$\rfloor$
\end{itemize}
Gli alberi binari heap non impongono un ordinamento totale, ma parziale. \
\begin{lstlisting}[caption= Memorizzazione Heap con Vettore]
% Vettore di memorizzazione
A[1 ... n]
% Radice
root() = 1
% Padre nodo i
p(i) = floor(i/2)
% Figlio sinistro nodo i
l(i) = 2i
% Figlio destro nodo i
r(i) = 2i + 1
\end{lstlisting}
\textbf{Algoritmi sulla struttura}: L'algoritmo più famoso sulla struttura Heap è l'algoritmo di ordinamento heapsort(). \\
\textbf{Codice heapSort}: È costituito da due parti: heapBuild() che si occupa di costruire un max-heap a partire da un vettore non ordinato e da maxHeapRestore() che si occupa di ripristinare le proprietà del max-heap ordinando nel mentre l'array.  \\
MaxHeapRestore() ha come parametri un vettore A, un indice i tale per cui gli alberi binari con radici l(i) e r(i) sono max-heap e la dimensione dim del vettore A. 
\begin{lstlisting}[ caption= maxHeapRestore]
maxHeapRestore(ITEM[] A, int i, int dim)
	int max = i
	if l(i) $\leq$ dim and A[l(i)] > A[max] then
		max = l(i)
	if r(i) $\leq$ dim and A[r(i)] > A[max] then
		max = r(i)
	if i $\neq$ max then
		A[i] $\leftrightarrow$ A[max]
		maxHeapRestore(A, max, dim)
\end{lstlisting}
HeapBuild() ha come parametri un vettore A e la sua dimensione n. 
\begin{lstlisting}[ caption= heapBuild]
heapBuild(ITEM[] A, int n)
	for i = floor(n/2) downto 1 do
		maxHeapRestore(A, i, n)
\end{lstlisting}
Usando gli algoritmi sopra definiti possiamo costruire una nuova funzione heapSort() che permette di ordinare un array A di dimensione n. 
\begin{lstlisting}[ caption= heapSort]
heapSort(ITEM[] A, int n)
heapBuild(A, n)
	for i = n downto 2 do
		A[1] $\leftrightarrow$ A[i]
		maxHeapRestore(A, 1, i-1)
\end{lstlisting}
\textbf{Complessità computazionale}: La chiamata di heapBuild() costa $\Theta$(\textit{n}), mentre la chiamata maxHeapRestore() costa $\Theta$(\textit{log i}) con i elementi nel heap. La maxHeapRestore() viene eseguita per un numero di volte che va da 2 a n e quindi il costo totale è $\Theta$(\textit{n log n}).

\subsection{Code a priorità}
\textbf{Descrizione}: Le code a priorità sono usa struttura dati astratta simile ad una coda in cui ogni elemento inserito possiede una priorità. Possono esistere min-priority queue in cui l'estrazione avviene per valori crescenti di priorità, mentre nelle max-priority queue l'estrazione avviene per priorità decrescente.\\
\textbf{Specifica}: interfaccia dei metodi per la gestione della coda a priorità.\
\begin{lstlisting}[ caption= Min Priority Queue Specifica]
% Crea una coda con priorita' vuota
void MinPriorityQueue()
% Restituisce true se la coda con priorita' e' vuota
boolean isEmpty()  
% Restituisce l'elemento minimo di una coda con priorita (non vuota)
Item min()  
% Rimuove e restituisce il minimo da una coda con priorita' (non vuota)
Item deleteMin()  
% Inserisce l'elemento x con priorita' p nella coda con priorita' e restituisce un oggetto PriorityItem che identifica x all'interno della coda
PriorityItem insert(Item x, int p)
% Diminuisce la priorita' dell'oggetto identificato da y portandola a p	
decrease(PriorityItem y, int p) 
\end{lstlisting}
Di seguito una possibile implementazione dei metodi della min-priority queue facente uso del min-heap.
\begin{lstlisting}[ caption= Min Priority Queue Implentazione]
% Definizione di PriorityItem
PriorityItem
	int priority       %  Priorita'
	Item value         % Elemento
	int pos           % Posizione nel vettore heap
	
% Funzione per scambiare due elementi nella coda a priorita'
swap(PriorityItem[] H, int i, int j)
	H[i] $\leftrightarrow$ H[j]
	H[i].pos = i
	H[j].pos = j
	
% Definizione della struttura della coda a priorita'
PriorityQueue
	int capacity 		% Numero massimo di elementi nella coda
	int dim 			% Numero attuale di elementi nella coda
	PriorityItem[] H    % Vettore heap

% Inizializzazione della coda a priorita'
PriorityQueue priorityQueue(int n)
	PriorityQueue t = new PriorityQueue
	t.capacity = n
	t.dim = 0
	t.H = new PriorityItem[1...n]
	return t
	
% Inserimento nella coda
PriorityItem insert(Item x, int p)
	precondition: dim < capacity
	
	dim = dim + 1
	H[dim] = new PriorityItem()
	H[dim].value = x
	H[dim].priority = p
	H[dim].pos = dim
	int i = dim
	while i > 1 and H[i].priority < H[p(i)].priority do
		swap(H, i, p(i))
		i = p(i)
	return H[i]
	
% Cancellazione dell'elemento con priorita' minore
Item deleteMin()
	precondition: dim > 0
	
	swap(H, 1, dim)
	dim = dim - 1
	minHeapRestore(H, 1, dim)
	return H[dim + 1].value

% Restituisce l'elemento con priorita' minore
Item min()
	precondition: dim > 0
	
	return H[1].value

% Decrementa la priorita' di un elemento della coda
decrease(PriorityItem x, int p)
	precondition: p < x.priority
	
	x.priority = p
	int i = x.pos
	while i > 1 and H[i].priority < H[p(i)].priority do
		swap(H, i, p(i))
		i = p(i)	
\end{lstlisting}

\textbf{Costo computazionale}: le operazioni di insert(), deleteMin(), decrease() che modificano  l'heap e lo sistemano hanno un costo di O(\textit{log n}), mentre min() ha un costo pari a $\Theta$(\textit{1}).\\\\

\subsection{Insiemi disgiunti - Merge Find Set}
\textbf{Motivazione e operazioni fondamentali}: Siamo interessati a gestire una collezione $S = {S1, S2, ...Sk}$ di insiemi dinamici disgiunti tali che: 
\begin{itemize}
	\item $\forall i,j : i \neq j \Rightarrow Si \cap Sj = \emptyset$ 
	\item $\bigcup\limits_{i=1}^{k} Si = Sj$, dove $k =|S|$
\end{itemize}
Le operazioni fondamentali possibili su questa struttura sono:
\begin{itemize}
	\item Creare n insiemi disgiunti, ognuno composto da un unico elemento
	\item merge(): unire più insiemi
	\item find(): identificare l'insieme a cui appartiene un elemento\\
\end{itemize}
\textbf{Rappresentante}: Ogni insieme è identificato da un rappresentante univoco. Il rappresentante dell'insieme $Si$ è un qualunque membro di $Si$. Più operazioni di ricerca del rappresentante svolte su uno stesso insieme devono restituire sempre lo stesso oggetto. Solo in caso di unione con altro insieme, il rappresentante degli insiemi può cambiare.\\\\
\textbf{Specifica MFSET}:
\begin{lstlisting}[ caption=Specifica MFSET]
%Crea n componenti {1}, ... ,{n}
MFSET Mfset(int n)
%Restituisce il rappresentante della componente contenente x
int find(int x)
%Unisce le componenti che contengono x e y
merge(int x, int y)
\end{lstlisting}
\textbf{\\Realizzazione}: Esistono 2 possibili tecniche per memorizzare gli insiemi disgiunti usando due strutture dati differenti.\\

\subsubsection{Realizzazione basata su insieme di liste}
\textbf{Descrizione}: Ogni insieme viene rappresentato da una lista concatenata. Il primo oggetto di una lista è il rappresentante dell'insieme e ogni elemento delle lista contiene: un intero/oggetto, un puntatore all'elemento successivo e un puntatore al rappresentante.\\
\textbf{Operazioni}:
\begin{itemize}
	\item Operazione find(x): viene restituito il rappresentante dell'insieme x. L'operazione di find(x) richiede tempo O(1).
	\item Operazione di merge(x,y): si "appende" la lista che contiene y alla lista che contiene x, modificando i puntatori ai rappresentanti nella lista "appesa" (per ogni elemento di y devo modificare il rappresentante). Nel caso pessimo, il costo di n operazione è O($n^{2}$) e quindi il coso ammortizzato di una operazione vale O(n). 
\end{itemize}
\subsubsection{Realizzazione basata su insieme di alberi (foresta)}
\textbf{Descrizione}: Ogni insieme viene rappresentato da un albero. La radice è il rappresentante dell'insieme e ha un puntatore a sé stessa (per indicare che è il quel nodo il rappresentante). Ogni nodo dell'albero contiene: un intero/oggetto, un puntatore al padre.\\
\textbf{Operazioni}:
\begin{itemize}
	\item Operazione find(x): risale la lista dei padri di x fino a trovare la radice e restituisce la radice stessa come rappresentante. Il costo è O(n) nel caso pessimo di albero lineare (in pratica una lista)
	\item Operazione di merge(x,y): si aggancia l'albero radicato in y all'albero alla radice dell'albero x modificando il puntatore al padre della radice di y. Il costo è O(1) (non consideriamo il costo del find per cercare la radice).  
\end{itemize}

\subsubsection{Euristiche per migliorare la realizzazione}
\textbf{Definizione algoritmo euristico}: particolare tipo di algoritmo progettato per risolvere il problema più velocemente, qualora i metodi classici siano troppo lenti o per trovare una soluzione approssimata qualora i metodi classi siano troppo lenti. Applicheremo delle euristiche ai metodi find e merge visti finito ad ora per migliorare la complessità delle due realizzazioni. \\\\
\textbf{Euristiche per MFSET}:
\begin{itemize}
	\item Euristica del peso (applicata alle liste): è una strategia per diminuire il costo dell'operazione merge. Consiste nel memorizzare nelle liste l'informazione sulla loro lunghezza e nel momento del merge agganciare la lista più corta a quella più lunga in modo da modificare meno rifermenti al rappresentante. La lunghezza della lista può essere salvata e acceduta in tempo costante. Tramite analisi ammortizzata il costo di n-1 operazioni di merge è $\mathcal{O}(n log n)$ e il costo delle singole operazioni è $\mathcal{O}(log n)$
	\item Euristica del rango (applicata agli alberi): ogni nodo mantiene informazioni sul proprio rango, cioè rank[x] del nodo x è il numero di archi del cammino più lungo tra x e una delle foglie che discendono da x (rango = altezza del sottoalbero associato al nodo). Il nostro scopo è usare il rango mantenere bassa l'altezza degli alberi. Se i due alberi hanno rango uguale si aggancia indifferentemente un albero alla radice dell'altro e il rango dell'albero finale cresce di 1. Se gli alberi hanno rango diverso si aggancia l'albero con rango più basso all'albero con rango più alto e l'altezza rimane inalterata. Possiamo dimostrare che (vedi slide) un albero MFSET con n nodi ha altezza inferiore a log n, quindi l'operazione find(x) ha costo   $\mathcal{O}(log n)$
	\item Euristica della compressione dei cammini (applicata agli alberi): ogni volta che viene eseguita la find su un particolare albero, l'albero viene "appiattivo" in modo che le successive find sullo stesso albero siano svolte in O(1) (l'appiattimento viene eseguito solo in caso di modifica dell'albero con la merge). Questo è possibile costruendo un albero in cui tutti i nodi siano abbiano come padre direttamente il rappresentante e l'albero abbia altezza 1.  
\end{itemize}
Applicando entrambe le euristiche sugli alberi il rango non è più l'altezza reale, ma un limite superiore all'altezza del nodo (il rango corretto sarebbe troppo costoso da mantenere). Il costo di m operazioni di find in un insieme di m elementi usando gli alberi con le varie euristiche è  $\mathcal{O}(m \cdot \alpha(n))$, con $\alpha(n)$ funzione inversa di Ackermann che cresce molto lentamente. Il costo ammortizzato di una singola operazione è quindi O(1). \\\\
\textbf{Implementazione}: Implementazione della struttura MFSET usando gli alberi e le varie euristiche presentante sopra.\
\begin{lstlisting}[ caption=Implementazione MFSET]
MFSET
	int[] parent
	int[] rank
	
MFSET Mfset(int n)
	MFSET t = new MFSET
	t.parent = int[1...n]
	t.rank = int[1...n]
	for i = 1 to n do
		t.parent[i] = i
		t.rank[i] = 0
	return t
	
merge(int x, int y)
	rx = find(x)
	ry = find(y)
	if rx != ry then
		if rank[rx] > rank[ry] then
			parent[ry] = rx
		else if rank[ry] > rank[rx] then
			parent[rx] = ry
		else
			parent[rx] = ry
			rank[ry] = rank[ry] + 1

int find(int x)
	if parent[x] != x then
		parent[x] = find(parent[x])
	return parent[x]
\end{lstlisting}
\textbf{\\Complessità}: Nella tabella sotto vengono riportati i costi degli algoritmi find e merge degli insieme disgiunti. I costi con l'asterisco (*) sono stati calcolati con l'analisi ammortizzata. I costi di merge non tengo costo del costo necessario ad eseguire le 2 find ma assumono di avere già i rappresentati dei due insiemi coinvolti.\\
\begin{center}
	\renewcommand{\arraystretch}{1.2}
	\begin{tabular}{ |c|c|c| } 
		\hline
			\textbf{Algoritmo} & find() & merge()\\ 
		\hline
			Liste & $\mathcal{O}$(1) &  $\mathcal{O}$($n$)\\ 
		\hline
			Alberi &  $\mathcal{O}$($n$) &  $\mathcal{O}$($n^{2}$) \\
		\hline
			Liste + Euristica sul peso & $\mathcal{O}$(1) &  $\mathcal{O}$($log n$)*\\
		\hline
			Alberi + Euristica sul rango & $\mathcal{O}$($log n$) &  $\mathcal{O}$(1)\\
		\hline
			Alberi + Euristica sul rango + Compressione cammini & $\mathcal{O}$(1)* &  $\mathcal{O}$(1)\\
		\hline
	\end{tabular}
\end{center}
\newpage
\subsubsection{Esempio applicazione MFSET - Componenti connesse dinamiche}
\textbf{Problema}: Questa struttura dati può essere trovare le componenti connesse di un grafo non orientato dinamico.\\\\
\textbf{Implementazione}:
\begin{lstlisting}[ caption=Componenti connesse con MFSET]
MFSET cc(Graph G)
	%n = numero di nodi del grafo
	MFSET M = Mfset(G.n)
	foreach u $\in$ G.V() do
		foreach v $\in$ G.adj(u) do
			M.merge(u,v)
	return M
\end{lstlisting} 
\textbf{\\Funzionamento}: si inizia con componenti connesse costituite da un unico vertice del grafo. Per ogni arco (u,v) $\in$ E, si esegue merge(u,v) per creare un unico insieme che contiene entrambi i lati dell'arco (visto che appartengono alla stessa componente connessa). Alla fine ogni insieme disgiunto rappresenta una componente connessa. Nodi appartenenti a componente connesse non saranno nello stesso insieme in quanto non ci sono archi che collegano i vari nodi su cui eseguire la merge.\\\\
\textbf{Complessità}: O(n) + m operazioni di merge(), che si solito ha costo O(1) costante e la complessità risultante sarebbe (On+m). Questo algoritmo è preferibile rispetto a quello visto fino ad ora quando il grafo che dobbiamo gestire è dinamico (possiamo aggiungere anche su quel grafo) in quando con questa versione basterebbe eseguire una merge() sull'arco aggiunto, mentre con la versione classica bisognerebbe eseguire da capo l'algoritmo. \\\\

\newpage
\section{Divide-et-impera}
\subsection{QuickSort}
\textbf{Problema}: Dato un vettore di ITEM A[1...n] e una coppia di indici start e end tali che 1  $\leq$ start  $\leq$ end  $\leq$ n, ordina il vettore A[i...n] in modo crescente. ITEM è un tipo di dato in cui è definito il confronto tra elementi.\\
\textbf{Soluzione}: l'algoritmo utilizza la tecnica divide-et-impera e si compone di due funzioni: la funzione pivot() si occupa, dato un sotto vettore, di definire un perno e di spostare a sinistra del perno tutti gli elementi più piccoli del perno stesso e  destra  gli elementi più grandi, mentre la funzione QuickSort() si occupa di richiamare la funzione in modo ricorsivo su sotto array di dimensione minore.\
\begin{lstlisting}[ caption=Quicksort]
int pivot(ITEM A[], int start, int end)
	ITEM P = A[start]
	int j = start
	for i = start+1 to end do
		if A[i]<p then 
			j = j+1
			A[i] $\leftrightarrow$ A[j]
	A[start] = A[j]
	A[j] = p	
	return j

void QuickSort(ITEM A[], int start, int end)
	if start < end then
		int j = pivot(A, start, end)
		QuickSort(A, start, j-1)
		QuickSort(A, j+1, end)		
\end{lstlisting}
\textbf{Costo computazionale}:
\begin{itemize}
 	\item Pivot: $\Theta$(\textit{n}) 
 	\item QuickSort: si comporta in modo diverso in base all'ordine degli elementi. Nel caso pessimo vale $\Theta$(\textit{n\textsuperscript{2}}), nel caso ottimo  $\Theta$(\textit{n log n}), mentre il caso medio è difficile da calcolare. 
\end{itemize}
\textbf{Specifica e assunzioni}: Metodi per la gestione della struttura. Assumiamo di memorizzare nella struttura degli interi e non degli oggetti e prevediamo che esista una associazione intero-oggetto memorizzata esternamente.\


\subsection{Gap}
\textbf{Problema}: Dato un vettore V di interi contenente n $\geq$ 2 elementi e tale che V[1] $<$ V[n], trovare la posizione di un gap nel vettore. In un vettore V contente almeno 2 interi, un gap è un indice i tale che 1 $<$ i $\leq$ n, tale che V[i-1] $<$ V[i]. Le condizioni poste nel problema di avere almeno 2 elementi e che valga V[1] $<$ V[n], garantiscono che ci sia almeno un gap.\\
\textbf{Soluzione}: implementazione ricorsiva con tecnica divide-et-impera. Uso una funzione wrapper per avere a disposizione più parametri.
\newpage
\begin{lstlisting}[ caption=Gap]
int gap(int[] V, int n)
	return gapRect(V, 1, n)
	

int gapRec(int[] V, int i, int j)
	if j == i+1 then
		return j
	m = floor((i+j)/2)
	if V[m] < V[j] the
		return gapRec(V, m, j)
	else 
		return gapRec(V, i, m)
\end{lstlisting}\

\end{document}
