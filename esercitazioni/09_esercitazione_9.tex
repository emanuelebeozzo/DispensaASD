\documentclass[../cheatSheetAlgoritmi.tex]{subfiles}
\begin{document}

\section{Esercitazione 9 - 30/03/20}
\textbf{Sciatori}\\
Siano dati n sciatori di altezza $p_{1}, ..., p_{n}$ ed $n$ paia di sci di lunghezza $s_{1}, ..., s_{n}$. Il problema è assegnare ad ogni sciatore un paio di sci, in modo da minimizzare la differenza totale fra le altezza degli sciatori e la lunghezza degli sci; ovvero, se allo sciatore $i$ è assegnato il paio di sci $h(i)$, minimizzare la seguente quantità:


\begin{equation}
	\sum_{i = 1}^{n}\mid{p_{i} - s_{h(i)}}\mid
\end{equation}
Si consideri il seguente algoritmo greedy. Si individui la coppia (sciatore, sci) con la minima differenza. Si assegni allo sciatore questo paio di sci. Si ripete con gli sciatori restanti fino a quando non si è terminato. Provare la correttezza di questo algoritmo o trovare un controesempio.

\bigskip
\textbf{Soluzione:} L'algoritmo proposto non è una soluzione corretta al problema proposto, è necessario pensare ad esempio al caso in cui $p = [8, 13]$ e $s = [10, 5]$. In questo caso la differenza minima è data associando $p = 8$ con $s = 10$ e di conseguenza $p = 13$ verrà associato con $s = 5$ per un totale di 10. Se invece avessimo associato $p = 8$ a $s = 5$ e $p = 13$ a $s = 10$ avremmo ottenuto come risultato 6.\\
Questa osservazione ci fornisce anche un'ottima base per la soluzione greedy dell'algoritmo per cui è necessario \textbf{ordinare entrambi i vettori} di sciatori e sci in modo da \textbf{associare sempre lo sciatore di altezza maggiore con il paio di sci di lunghezza maggiore}.

\bigskip  
\textbf{Sfilatino alla Nutella - Compito 31/05/13}\\
Nella sagra di Hateville, è stato realizzato lo sfilatino alla Nutella più lungo del mondo, lungo $L$ centimetri. Ora si tratta di realizzare un altro record: il maggior numero di persone servite con lo stesso sfilatino. Alla sagra sono presenti $n$ persone, dove la persona $i$-esima chiede un segmento di sfilatino lungo $V[i]$ centimetri; secondo il regolamento, ogni richiesta va servita esattamente, ovvero se la persona $i$ verrà servita, riceverà il segmento richiesto. Tutte le lunghezze sono intere. Scrivere un algoritmo che restituisca il numero massimo di persone che possono essere servite con lo sfilatino. Non è necessario utilizzare tutto lo sfilatino. Oltre a calcolare la complessità dell'algoritmo proposto, discutere anche la correttezza, menzionando la tecnica scelta e specificando bene perché tale tecnica può essere applicata in questo caso.
\begin{lstlisting}[caption=Sfilatino alla Nutella]
int maxSquare(int[] V, int n, int L)
	% Ordino le richieste per dimensione
	int served = 0
	for i = 1 to n and L - V[i] $\geq$ 0 do
		served = served + 1
		L = L - V[i]
	return served
\end{lstlisting}
Nonostante scorrere il vettore delle richieste richieda tempo lineare l'algoritmo è limitato superiormente dall'ordinamento del vettore delle richieste, cioè pari a $\mathcal{O}(n \log{n})$.\\
Supponiamo sia $S[i...n]$ il set di richieste soddisfatte in maniera ottimale e la cui cardinalità è la maggiore possibile. Sia $j$ lo sfilatino con dimensione minore in $V[1...n]$; possono accadere 3 casi:
\begin{itemize}
	\item $j$ si trova in prima posizione in $S[i...n]$, dunque la soluzione è comunque ottimale.
	\item $j$ non si trova in prima posizione (ma sempre all'interno di $S[i...n]$), dunque è possibile trovare una soluzione equivalente $S'[j...n]$ dove $j$ si trova in prima posizione in quanto scambiato con $S[i]$.
	\item $j$ non si trova all'interno dell'insieme considerato, ma quindi è possibile selezionare lo sfilatino con dimensione minore all'interno di $S[i...n]$ e sostituirlo con $j$ ottenendo quindi due insiemi dove $\sum_{k \in S} V[k] \geq \sum_{k \in S'} V[k]$ dove magari in $S'$ sarebbe possibile aggiungere una nuova richiesta ed aumentarne la cardinalità ma ciò è impossibile perché $S$ ha la cardinalità maggiore possibile.
\end{itemize}
\textbf{High Line - Compito 29/06/17}\\
Funzionamento: \\
Ordiamo gli intervalli per limite inferiore di copertura, consideriamo il primo e salviamo dunque il valore minimo e il valore massimo coperto da tale intervallo nelle variabili inf e sup rispettivamente. Il primo elemento per definizione è incluso in quanto quello che parte dal punto più basso (ovviamente può essere sostituito da un intervallo che parte nello stesso punto e termina successivamente)\\
Sono gestiti i seguenti casi: 
\begin{itemize}
    \item Se l'estremo inferiore non include 0 allora non siamo riusciti a coprire una parte del tracciato, l'algoritmo deve dunque restituire -1
    \item $\forall i \mid 2 \leq i \leq n \land sup < L$ (irrigatore) si controlla:
    \begin{itemize}
        \item $D[i] + R[i] > sup$, l'intervallo considerato supera l'ultimo conteggiato, abbiamo dunque 3 casi:
        \begin{itemize}
            \item $inf == D[i] - R[i]$, l'intervallo considerato inizia nello stesso punto di quello conteggiato, è possibile dunque considerare $sup = D[i] + R[i]$.
            \item $sup < D[i] - R[i]$, l'intervallo considerato inizia dopo del termine di quello conteggiato, non si riesce a coprire l'intervallo ($sup, D[i] - R[i]$), l'algoritmo deve restituire -1
            \item Altrimenti, ovvero nel caso in cui $D[i] + R[i] > sup$ e $sup > D[i] - R[i]$; sono dunque costretto a considerare un ulteriore irrigatore aggiornando il conteggio e gli estremi di inizio e fine.
        \end{itemize}
    \end{itemize}
    \item Se al termine dell'analisi non si è coperto l'intero percorso si ritorna -1
\end{itemize}
\begin{lstlisting}[caption=HighLine]
int HighLine(int[] D, int[] R, int n, int L)
    mergeSort(D, V, n) % Ordinamento per D[i] - R[i] in maniera crescente
    int count = 0
    int sup = D[1] + R[1]
    int inf = D[1] - R[1]
    if inf > 0 then
        count = -1
    int i = 2 
    while i $\leq$ n and sup $<$ L and count $\geq$ 0 do 
        if D[i] + R[i] > sup then
            if inf $==$ D[i] - R[i] then
                sup = D[i] + R[i]
            else if sup < D[i] - R[i] then
                count = -1
            else 
                count = count + 1
                inf = D[i] - R[i]
                sup = D[i] + R[i]
        i = i + 1
    if sup < L then
        count = -1
    return count
\end{lstlisting}
La complessità computazionale dell'algoritmo proposto è $\mathcal{O}(n\log n)$, dovuta all'ordinamento per copertura iniziale in maniera crescente. È infatti sufficiente uno scorrimento di costo $\mathcal{O}(n)$ per il calcolo della soluzione.
La correttezza può essere dimostrata nel seguente modo: \\
\textbf{Sottostruttura ottima} \\
Sia $S = \{i, ... n\}$ una soluzione ottima, $i \in S$ un elemento di $S$. Si ha che $S - \{i\}$ è una soluzione ottima per il sotto-problema $[0, L] - [D[i] - R[i], D[i] + R[i]]$. Se così non fosse allora esisterebbe una soluzione $S' = \{i', ... n'\}$ tale per cui la cardinalità $\mid S' \mid$ $<$ $\mid S - \{i\} \mid$, ma questo è impossibile in quanto $\mid S' \cup \{i\} \mid$ $<$ $\mid S \mid$ rendendo di fatto tale insieme una soluzione ottima per il problema originale violando l'assunzione che S sia ottimale.\\
\textbf{Proprietà greedy} \\
Sia $S = \{i, ... n\}$ l'insieme composto dal minor numero di irrigatori tali da coprire lo spazio da 0 a $L$ metri. Sia dunque $j$ l'intervallo con limite inferiore di copertura più basso e, a parità di copertura iniziale, sia quello con maggior lunghezza di copertura finale. Si possono verificare i seguenti casi:
\begin{itemize}
    \item $j$ si trova in prima posizione, la soluzione è comunque ottima.
    \item $j$ non si trova in prima posizione, dunque la soluzione $S' = \{j, ..., i, ... n\}$ ottenuta scambiando l'elemento $i$ (in prima posizione nell'insieme $S$) e l'elemento j, ha la stessa cardinalità dell'insieme $S$ ed è ottima in quanto è una permutazione dell'insieme originale.
    \item $j$ non è compresa in $S$; sia $S' = S - \{i\} \cup \{j\}$ l'insieme ottenuto rimuovendo l'elemento $i$ tale che il suo limite inferiore sia maggiore o uguale a quello di $j$ e il suo limite superiore sia minore o al più uguale a quello di $j$. Si noti che tale $i$ esiste in quanto $j$ per definizione è l'intervallo che ha il limite inferiore minore e a parità di questo è anche quello che gode di una maggiore limite superiore. L'intervallo $S'$ ha la stessa cardinalità dell'intervallo $S$, rendendo di fatto $S'$ una soluzione ottima.
\end{itemize}
 
\end{document}