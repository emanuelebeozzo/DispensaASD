\documentclass[../cheatSheetAlgoritmi.tex]{subfiles}
\begin{document}

\subsection{Esercitazione 8 - 16/03/20}
\textbf{Massima Copertura - Compito 05/06/2014}\\
Si considerino n segmenti sulla retta delle ascisse, dove l'i-esimo segmento inizia nella coordinata $a[i]$ (inclusa) e termina nella coordinata $b[i]$ (esclusa). Scrivere un algoritmo che prenda in input i vettori $a$, $b$ e la dimensione $n$, e restituisca il sottoinsieme di segmenti indipendenti (che non si intersecano) di copertura massimale, ovvero che copre la maggior parte della retta delle ascisse. Valutare il costo computazionale dell'algoritmo proposto
\begin{lstlisting}[caption=Massima copertura]
int massimaCopertura(int[] a, int[] b, int n)
    int[] w = new int[1...n]
    for i = 1 to n do
        w[i] = b[i] - a[i]
    return maxInterval(a, b, w, n)
\end{lstlisting}

La complessità computazionale è $\mathcal{O}(n \log n)$, dovuta alla funzione maxInterval. 

È possibile osservare infatti che il problema proposto può essere trattato come il problema \hfill \break dell'insieme indipendente di intervalli di peso massimo, il cui peso è dato dalla grandezza dell'intervallo perché cerchiamo il sottoinsieme di intervalli disgiunti più grande possibile.
 
Per l'algoritmo maxInterval è sufficiente consultare la sezione \emph{Insieme indipendente di intervalli pesati} nel capitolo \emph{Programmazione dinamica}.

\bigskip
\textbf{I Promessi Sposi}\\
Scrivere un algoritmo che prenda in input una stringa testo $T$ di $n$ caratteri e una stringa pattern $P$ di $m$ caratteri e restituisca il numero di volte distinte che la stringa $P$ appare come sotto-sequenza di $T$. Discutere correttezza e complessità dell'algoritmo.
\begin{lstlisting}[caption=Sottosequenze di un pattern in un testo (Promessi Sposi)]
int promessiSposi(string[] T, int n, string[] P, int m)
	int[][] DP = new int[0...n][0...m]
	for i = 0 to n do
		for j = 0 to m do
			if i $==$ 0 then
				DP[i][j] = 0
			else if j $==$ 0 then
				DP[i][j] = 1
			else 
				DP[i][j] = -1
	return promessiSposiRec(T, P, DP, n, m)
	
int promessiSposiRec(string[] T, string[] P, int[][] DP, int i, int j)
	if DP[i][j] $<$ 0 then
		if T[i] $==$ P[j] then
			DP[i][j] = promessiSposiRec(T, P, DP, i-1, j-1) + promessiSposiRec(T, P, DP, i-1, j)
		else
			DP[i][j] = promessiSposiRec(T, P, DP, i-1, j)
	return DP[i][j]
\end{lstlisting}
\newpage
La complessità dell'algoritmo è $\mathcal{O}(nm)$ mentre è possibile scrivere la sua equazione di ricorrenza nel seguente modo:
\begin{equation*}
  	DP[i][j] =\begin{cases}
    	0 & \text{$i = 0$}\\
    	1 & \text{$j = 0$}\\
    	DP[i-1][j-1] + DP[i-1][j] & \text{$T[i] == P[j]$}\\
    	DP[i-1][j] & \text{$altrimenti$}
  	\end{cases}
\end{equation*}
$DP[i][j]$ rappresenta il numero di sotto-sequenze del prefisso del pattern $P_{j}$ (cioè nel pattern in $P[1...j]$) contenute nel prefisso del testo $T_{i}$, dunque avremo in $DP[n][m]$ le occorrenze totali di P nel testo T. I casi base sono rappresentati da:
\begin{itemize}
	\item $i = 0$, cioè è stato analizzato tutto il testo ma non il pattern e non è stata trovata dunque un occorrenza completa. Il valore di ritorno è 0.
	\item $j = 0$, cioè è stato analizzato tutto il pattern ed è quindi stata trovata un occorrenza completa. Il valore di ritorno è 1.
	\item $T[i] == P[j]$, cioè sono stati trovati due caratteri che sono uguali nel pattern e nel testo. In questo caso è necessario far avanzare sia il pattern che il testo in modo da verificare che sia possibile avere un match completo, sia cercare altre possibili occorrenze con quella medesima lettera.
	\item $T[i] \neq P[j]$, si passa alla prossima lettera del testo e si prosegue con la ricerca del pattern.
\end{itemize}

\bigskip
\textbf{D20 - Compito 02/09/13}\\
Siano dati $n$ dadi, con il dado i-esimo dotato di $F[i]$ facce numerate da $1$ a $F[i]$. Scrivere un algoritmo che restituisca il numero di modi diversi con cui è possibile ottenere una certa somma $X$ sommando i valori di tutti i dadi. Permutazioni: contatele oppure no, ma siatene consapevoli...
\begin{lstlisting}[caption= D20 con permutazioni]
int D20(int F[], int n, int X)
    int[][] DP = new int[1..n][1..X]
    for i = 1 to n do
        for j = 1 to X do
            DP[i][j] = -1
    return D20Rec(F, DP, n, X)

int D20Rec(int[] F, int[][] DP, int i, int x)
    if i $==$ 0 and x $==$ 0 then
        return 1
    else if x < 0 or i $==$ 0 then
        return 0
    else
        if DP[i][j] < 0 then
            DP[i][j] = 0
            for f = 1 to F[i] do
                DP[i][j] = DP[i][j] + D20Rec(F, DP, i - 1, x - f)
        return DP[i][j]
\end{lstlisting}
La complessità computazionale della soluzione presentata è $\mathcal{O}(nX)$, ovvero pseudopolinomiale; in particolare siano $k$ il numero di bit necessari per rappresentare il numero $X$, vale dunque che $k = \log(X)$ e la complessità si traduce in $\mathcal{O}(n2^k)$. 
La correttezza della soluzione è rappresentata dalla seguente formula ricorsiva: \\
Sia $DP[i][x]$ il numero di combinazioni con cui è possibile rappresentare il numero x avendo a disposizione i dadi ciascuno di facce differenti, rappresentate dal vettore F. La soluzione può essere calcolata come segue:
\begin{equation*}
    DP[i][x]=\begin{cases}
        1 & \text{$i == 0$ \textbf{and} $x == 0$}\\
        0 & \text{$x < 0$} \\
        \sum_{\forall j \in F[i]} DP[i-1][x - j] & \text{altrimenti}
    \end{cases}
\end{equation*}
Dove i casi presi in considerazione sono i seguenti:
\begin{itemize}
    \item Il valore di combinazioni possibili con 0 dadi e $x = 0$ è 1, in quanto non ho pù dadi disponibili e sono riuscito a formare il numero desiderato con quello che avevo a disposizione.
    \item Il valore di combinazioni possibili per ottenere un numero $x$ tale che $x < 0$ è 0, in quanto ho superato il limite imposto dal numero stesso.
    \item Se possiedo dadi disponibili allora posso formare il numero $x$ considerando le combinazioni possibili con un dado in meno e ottenendo un numero dal lancio che compensi a formare il numero $x$ cercato. In parole più semplici, sia $n$ il dado lanciato nell'istante corrente, il numero di combinazioni possibili per la quale è possibile ottenere $x$ è dato dalla somma delle combinazioni del lancio del dado $n-1$-esimo che contribuiscono alla composizione del numero $x$:
    $DP[n][x]$ = $\sum_{\forall j \in F[i]} DP[i-1][x - j]$
\end{itemize}
Tale soluzione restituisce dunque anche le permutazioni di valori che contribuiscono alla composizione del numero $X$.

\bigskip
Una soluzione alternativa per evitare il conteggio delle permutazioni potrebbe essere la seguente:
\begin{lstlisting}[caption= D20 senza permutazioni]
int D20(int F[], int n, int X)
    int M = max(F)
    int[][][] DP = new int[1..n][1..X][1..M]
    for i = 1 to n do
        for j = 1 to X do
            for m = 1 to M do
                DP[i][j][m] = -1
    return D20Rec(F, DP, n, X, 1)

int D20Rec(int[] F, int[][][] DP, int i, int x, int last)
    if i $==$ 0 and x $==$ 0 then
        return 1
    else if x < 0 or i $==$ 0 then
        return 0
    else
        if DP[i][j][last] < 0 then
            DP[i][j][last] = 0
            for f = last to F[i] do
                DP[i][j][last] = DP[i][j][last] + D20Rec(F, DP, i - 1, x - f, f)
        return DP[i][j][last]
\end{lstlisting}
In questo caso si fa utilizzo di una colonna in più all'interno della matrice delle soluzioni, in questo modo è possibile imporre un vincolo nella scelta del numero da estrarre dal dado tirato; infatti, estraendo solamente valori crescenti o al più uguali a quello considerato, non si conteggiano i doppioni. Un esempio potrebbe essere il seguente: supponiamo che il numero da ottenere sia 7 e che i dadi a disposizione siano 3 e abbiano 6 facce, per brevità assumiamo presenti solamente i seguenti insiemi: \\
$[1, 4, 2]$ $[1, 2, 4]$  $[2, 4, 1]$ \\
I quali sono la permutazione di uno stesso insieme formato dai numeri: 4, 1 e 2. Con il vincolo appena posto si fa in modo di prendere esclusivamente valori crescenti o uguali a quello selezionato, quindi tra questi insiemi il nostro algoritmo selezionerà solamente $[1, 2, 4]$. \\
La complessità della soluzione proposta è la $\mathcal{O}(nM^2X)$. Il fattore quadrato di M deriva dal fatto che per ogni cella è riempita costo $\mathcal{O}(M)$. \\
Sia $DP[i][x][m]$ il numero di combinazioni con cui è possibile rappresentare il numero x avendo a disposizione i dadi ciascuno di facce differenti, rappresentate dal vettore F, il cui numero è vincolato inferiormente dal valore m. \\
L'equazione di ricorrenza associata all'algoritmo è la seguente: 
\begin{equation*}
    DP[i][x][m]=\begin{cases}
        1 & \text{$i == 0$ \textbf{and} $x == 0$}\\
        0 & \text{$x < 0$} \\
        \sum_{j = m}^{F[i]} DP[i-1][x - j][m] & \text{altrimenti}
    \end{cases}
\end{equation*} \\
\textbf{Costo Partizione di un Vettore - Compito 05/06/14}\\
Scrivere un algoritmo che prenda in input un vettore $V$ contenente $n$ interi e un intero $k$ tale che $2 \leq k \leq n$ e restituisca il costo della k-partizione di $V$ di costo minimo. Il costo totale di tutte le partizioni è dato dalla partizione con il costo interno più alto ottenuto sommando i numeri del vettore appartenenti alla partizione in questione.\\
Esempio: $V = [2,3,7,-7,15,2]$, $k = 3$\\
$[2,3,7]$, $[-7,15]$, $[2]$ costo $2 + 3 + 7 = 12$\\
$[2,3]$, $[7]$, $[-7,15,2]$	costo $- 7 + 15 + 2 = 10$
\begin{enumerate}
	\item Soluzione per $k= 2$
	\item Soluzione per $k= 3$
	\item Soluzione generale
\end{enumerate}
Per la soluzione per $k = 2$ è possibile creare un vettore $DP$ contenente le somme degli elementi da $1$ a $i$ ($1 \leq i \leq n$) in modo da avere in un colpo solo il valore della partizione di sinistra (contenuta in posizione i-esima) e quella della partizione di destra (semplicemente calcolando il valore $n - i$).
\begin{lstlisting}[caption=2-partizioni]
int 2partition(int[] V, int n)
	int[] DP = new int[1...n]
	DP[1] = V[1]
	for i = 2 to n do
		DP[i] = V[i] + DP[i-1]
	int sx = 0
	int dx = 0
	int tmpMin = $\infty$
	int indexMin = -1
	for i = 1 to n do
		sx = DP[i]
		dx = DP[n] - DP[i]
		if max(sx, dx) < tmpMin then
				tmpMin = max(sx, dx)
				indexMin = i
	return indexMin
\end{lstlisting}
La complessità di questa soluzione è ovviamente $\mathcal{O}(n)$.\\
Nella soluzione con $k=3$ è sempre possibile creare un vettore $DP$: a differenza del caso precedente è necessario usare due cicli per fissare gli indici a cui verrà spezzato il vettore e applicare un ragionamento simile a quello fatto in precedenza
\newpage
\begin{lstlisting}[caption=3-partizioni]
int 3partition(int[] V, int n)
	int[] DP = new int[1...n]
	DP[1] = V[1]
	for i = 2 to n do
		DP[i] = V[i] + DP[i-1]
	int sx = 0
	int dx = 0
	int mid = 0
	int tmpMin = $\infty$
	int indexMin = -1
	for i = 1 to n do
		for j = i to n do
			sx = DP[i]
			mid = DP[j] - DP[i]
			dx = DP[n] - DP[j]
			if max(sx, mid, dx) < tmpMin then
					tmpMin = max(sx, mid, dx)
					indexMin = i
	return indexMin
\end{lstlisting}
La complessità di questa soluzione è $\mathcal{O}(n^{2})$.

Nonostante sembrerebbe la soluzione migliore, nel caso di $k$ partizioni non è affatto consigliato utilizzare $k-1$ cicli per suddividere il vettore. In questo caso la risoluzione più efficiente consiste nell'utilizzare l'algoritmo per quello che è definito \emph{Painter Partition Problem}. Per evitare incomprensioni con le soluzioni precedenti è necessario sottolineare che il vettore DP usato in precedenza verrà rinominato in $tot[1...n]$ e che $DP$ in questo caso farà riferimento ad una matrice.

\begin{lstlisting}[caption=k-partizioni]
int k_partition(int[] V, int n, int k)
	int[] tot = new int[1...n]
	tot[1] = V[1]
	for i = 2 to n do
		tot[i] = V[i] + tot[i-1]
	int[][] DP = new int[1...n][1...k]
	for i = 1 to n do
		for j = 1 to n do
			DP[i][j] = -1
	return partitionRec(V, tot, DP, n, k)
	
int partitionRec(int[] V, int[] tot, int[][] DP, int i, int t)
	if t > i then
		return +$\infty$
	else if t $==$ 1
		return tot[i]
	else
		if DP[i][t] < 0 then
			DP[i][j] = +$\infty$
			int cost = 0
			for j = 1 to i - 1 do
				cost = max(partitionRec(V, tot, DP, j, t - 1), tot[i] - tot[j])
				DP[i][t] = min(DP[i][t], cost)
		return DP[i][j]
\end{lstlisting}
\newpage
\noindent
Sia $DP[i][t]$ il minimo costo associato al sotto-problema di trovare la migliore t-partizione nel vettore $V[1...i]$. Il problema iniziale corrisponde a $DP[n][k]$ – ovvero trovare la migliore k-partizione in $V[1...i]$.
\begin{equation*}
    DP[i][t]=\begin{cases}
        +\infty & \text{$t > i$}\\
        tot[i] & \text{$t == 1$} \\
        min_{1 \leq j < i}max(DP[j][t-1], tot[i] - tot[j]) & \text{altrimenti}
    \end{cases}
\end{equation*}
\\La tabella ha dimensione $n \times k$, ogni cella richiede tempo $\mathcal{O}(n)$ per essere riempita e il costo computazionale è quindi $\mathcal{O}(kn^{2})$.

\bigskip
\textbf{Quadrato Binario - Compito 10/09/12}\\
Sia $A$ una matrice $n \times n$ di valori booleani $0/1$. Scrivere un algoritmo che prenda in input la matrice $A$ e la sua dimensione $n$, e restituisca la dimensione del più grande quadrato composto da valori $1$ contenuto nella matrice.\\
La soluzione proposta consiste nel memorizzare innanzitutto in una matrice $DP$ la matrice iniziale in modo da poterci lavorare successivamente. L'algoritmo si basa sul controllo del valore contenuto nella cella a sinistra, in alto, in alto a sinistra e della cella stessa che si sta esaminando; viene poi selezionato il valore minimo fra questi quattro e viene sommato uno. Questa condizione stabilisce che si formerà un quadrato di celle più grandi solo e soltanto se il valore delle quattro celle è uguale e quindi se di fatto tutte le celle precedenti contengono un quadrato di quella dimensione.
\begin{lstlisting}[caption=Quadrato Binario]
int maxSquare(boolean[][] A, int n)
	int[][] DP = new int[1...n][1...n]
	int maxValue = $-\infty$
	for i = 1 to n do
		for j = 1 to n do
			DP[i][j] = A[i][j]
	for i = 2 to n do
		for j = 1 to n do
			if DP[i][j] $\neq$ 0 then
				DP[i][j] = min(DP[i][j], DP[i-1][j-1], DP[i-1][j], DP[i][j-1])+1
				maxValue = max(DP[i][j], maxValue)
return maxValue
\end{lstlisting}
La complessità dell'algoritmo proposto è pari a $\mathcal{O}(n^{2})$ ed è possibile scrivere la formula ricorsiva nel seguente modo:
\begin{equation*}
    DP[i][j]=\begin{cases}
        0 & \text{$A[i][j] = 0$}\\
        1 & \text{$A[i][j] = 1 \land (i = 1 \lor j = 1)$} \\
        min\{DP[i-1][j], DP[i][j-1], DP[i-1][j-1]\}+1 & \text{altrimenti}
    \end{cases}
\end{equation*}
\newpage
\end{document}