\documentclass[../cheatSheetAlgoritmi.tex]{subfiles}
\begin{document}

\section{Esercitazione 6 - 20/02/20}
\textbf{Ottimizza la somma}\\
Supponete di avere in input un vettore di n interi positivi distinti $V[1. . . n]$ e un valore $W$. Scrivere un algoritmo che:
\begin{enumerate}
	\item restituisca il massimo valore $X =  \sum_{i = 1}^{n} x[i] V[i]$ tale che $X \leq W$ e ogni $x[i]$ è un intero non negativo
	\item stampi il vettore $x$
\end{enumerate}

Ad esempio, per $V = [18,3,21,9,12,24]$ e $W = 17$, una possibile soluzione ottima è 
\break 
$x = [0,2,0,1,0,0]$ da cui deriva $X= 15$.

Discutere correttezza e complessità.

\bigskip
Da un'osservazione più attenta del problema ci si accorge che non è altro che una versione modificata di Knapsack (versione Senza Limiti) dove $w$ e $p$, rispettivamente i vettori che indicano il profitto e il peso di un oggetto, sono entrambi rappresentati da $V$. Nonostante bastasse richiamare l'algoritmo ricorsivo proposto nella parte di memoization proponiamo una versione iterativa che permette di stampare anche il vettore $x$:


\begin{lstlisting}[caption=Ottimizza la Somma]
int knapsack(int[] V, int n, int W)
	int x[] = new int[1...n]
	DP[] = new int[0...n][0...W]
	for i = 0 to n do
		DP[i][0] = 0
	for w = 0 to W do
		DP[0][w]
	for i = 1 to n do
		x[i] = 0
	for i = 1 to n do
		for w = 1 to W do
			if V[i] $\leq$ w then
				DP[i][w] = max(DP[i][w - V[i]] + V[i], DP[i-1][w])
			else
				DP[i][w] = DP[i-1][w]
	solution(DP, n, W, V, x)
	print("[")
	for i = 1 to n do
		print(x[i])
		print(" ")
	print("]")
	return DP[n][W]
	
solution(int[][] DP, int n, int W, int[] V, int[] x)
	if n > 0 and W > 0 then
		if DP[n][W] = DP[n-1][W] then
			solution(DP, n-1, W, V, x)
		else
			x[n]++
			solution(DP, n, W-1, V, x)
\end{lstlisting}
La complessità in questo caso risulta essere pari a $\mathcal{O}(nW)$ per il solito algoritmo iterativo di knapsack (pseudopolinomiale) e $\mathcal{O}(n + W)$. La correttezza può essere dimostrata in questo modo: sia $DP[i][w]$ la somma massima dei primi $i$ elementi (presi un numero indefinito di volte) limitata superiormente da $w$, vale la seguente equazione di ricorrenza
\begin{equation*}
  	DP[i][w] =\begin{cases}
    	0 & \text{$i = 0$  or  $w = 0$}\\
    	-\infty & \text{$w < 0$}\\
    	max(DP[i][c-V[i]]+V[i], DP[i-1][w]) & \text{$altrimenti$}
  	\end{cases}
\end{equation*}

\bigskip

\end{document}