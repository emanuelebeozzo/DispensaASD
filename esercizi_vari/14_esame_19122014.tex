\documentclass[../cheatSheetAlgoritmi.tex]{subfiles}
\begin{document}

\subsection{Esame 19/12/2014}
\textbf{Numero minimo di intervalli unitari} \\
Dato un vettore $V$ di $n$ numeri reali, bisogna calcolare il numero minimo di intervalli di lunghezza \emph{unitaria} in grado di ricoprire tutti i valori di $V$.  Un intervallo di lunghezza unitaria è definito da una coppia $[x, x + 1]$, estremi inclusi. Ad esempio se $n= 5$ e $V=\{2.5,3.8,1.5,3.1,1.8\}$, allora il numero minimo di intervalli unitari è $2$ e una possibile soluzione è $[1.5,2.5] ,[3,4]$. Scrivere un algoritmo 
\begin{center}
	\emph{int minIntervals(real[] V, int n)}
\end{center}
che risolva il problema. \\
\textbf{Soluzione} \\
Per risolvere il problema è possibile usare un approccio greedy. \\ 
Ordiniamo il vettore $V$ con un algoritmo di ordinamento basato su confronti come mergeSort (di complessità computazionale $\mathcal{O}(n \log n)$. \\
Vogliamo dimostrare che un intervallo l'intervallo $[V[1], V[1] + 1]$ (l'intervallo unitario che inizia in $V[1]$) fa sempre parte di una soluzione ottima.  \\
Si prenda una soluzione ottima. Se si consideri l'intervallo $[x, x + 1] \in S$ che "ricopre" $V[1]$, ovvero tale per cui $x \leq V[1] \leq x + 1$; tale intervallo deve esistere, in quanto $V[1]$ deve essere ricoperto. Poichè $V[1] \geq x$ e $V[1]$ è il primo punto del vettore, è possibile ottenere una soluzione $C - \{[x, x + 1]\} \cup \{[V[1], V[1] + 1]\}$ che ha la stessa dimensione di $C$, ricopre $V[1]$ e tutti i punti precedentemente ricoperti da $[x, x + 1]$. \\
In parole più semplici, pensiamo di avere il vettore $V$ ordinato; siamo consapevoli che tutto il vettore deve venire coperto, e in quanto $V[1]$ è il valore minore possibile che deve essere compreso, non possiamo fare altro che utilizzare un intervallo che parta da $V[1]$ e termini in $V[1] + 1$. Sia $j$ l'elemento che stiamo analizzando e sia $V[i]$ l'inizio dell'ultimo intervallo considerato nella soluzione ottimale; procedendo con la scansione possono verificarsi due casi:
\begin{itemize}
	\item $V[j] \leq V[i] + 1$, questo significa che il valore $V[j]$ è compreso nell'ultimo intervallo inserito nella soluzione ottimale
	\item $V[i] > V[i] + 1$, questo vuol dire che l'ultimo intervallo inserito nella soluzione ottimale non è sufficiente a ricoprire $V[j]$, dunque è necessario inserire un nuovo intervallo che abbia inizio in $V[j]$ all'interno della soluzione
\end{itemize}
L'algoritmo il cui funzionamento è stato discusso poc'anzi è il seguente:
\begin{lstlisting}[ caption = Numero minimo di intervalli unitari]
int minIntervals(real[] V, int n)
	mergeSort(V, n) % Ordinamento per valori crescenti
	int intervalli = 1
	int last = 1
	for i = 2 to n do
		if $V[i] > V[last] + 1$ then
			intervalli = intervalli + 1
			last = i
	return intervalli
\end{lstlisting}
La complessità computazionale di questa soluzione greedy è $\mathcal{O}(n \log n)$ dovuta all'ordinamento del vettore V di valori reali.
\newpage
\end{document}