\documentclass[../cheatSheetAlgoritmi.tex]{subfiles}
\begin{document}

\subsection{Esame 21/07/2014}
\textbf{Esercizio B4}\\
Sia data una stringa $s[1...n]$ di $n$ caratteri. Scrivere un algoritmo che restituisce la lunghezza della \emph{sottosequenza palindroma massimale} contenuta all’interno di s. Ricordiamo che una $sottosequenza$ è un sottoinsieme ordinato dei caratteri di $s$, anche non contigui. Ricordiamo che una stringa $palindroma$ si legge allo stesso modo da sinistra a destra e da destra a sinistra. Per $massimale$, si intende che non esistono sottosequenze palindrome più lunghe (ma possono esisterne altre della stessa lunghezza). Ad esempio, se l'input è “BBABCBCAB”, allora l’output deve essere 7 in quanto “BABCBAB” è la più lunga sottosequenza palindroma contenuta in essa. “BBBBB” e “BBCBB” sono anch’esse sottosequenze palindrome, ma non sono massimali. Discutere informalmente la correttezza della soluzione proposta e calcolare la complessità computazionale. Opzionale: scrivere un'estensione all'algoritmo che stampi una sottosequenza massimale.\\
\textbf{Soluzione:}\\
Il problema può essere risolto mediante programmazione dinamica utilizzando una matrice $DP[i][j]$ i cui indici sono utilizzati per analizzare due elementi della stringa $s$ in contemporanea. I casi base sono relativamente intuitivi e cioè che nel caso di una stringa vuota ritorneremo 0, per definizione un solo carattere di una stringa è palindromo dunque nel caso di una stringa contenente soltanto caratteri differenti dovremo ritornare 1 in quanto ogni lettera è palindroma con sè stessa, se i due caratteri considerati sono uguali dovremo inserirli mentre in caso contrario dovremmo spostarci muovendo o l'inidice iniziale o l'indice finale e selezionare il massimo una volta considerato quale caso sia più conveniente. Tutto ciò può essere riassunto dalla seguente equazione di ricorrenza
\begin{equation*}
  	DP[i][j]=\begin{cases}
  		0 & \text{$j < i$}\\
  		1 & \text{$j = i$}\\
  		DP[i+1][j-1] + 2 &\text{$s[i] = s[j]$}\\
  		max(DP[i+1][j], DP[i][j-1]) & \text{$s[i] \neq s[j]$}
  	\end{cases}
\end{equation*}
Questa può essere utilizzata per scrivere il seguente codice tramite memoization
\begin{lstlisting}[caption=Sottostringa Palindroma Massimale]
int maxPalindromeSubsequence(string s, int n)
	int[][] DP = new int[1...n][1...n]
	for i = 1 to n do
		for j = 1 to n do
			DP[i][j] = -1
	return maxPalindromeSubsequence(s, DP, 1, n)
	
int maxPalindromeSubsequence(string s, int[][] DP, int i, int j)
	if j $<$ i then
		return 0
	if j $==$ i then
		return 1
	if DP[i][j] < 0 then
		if s[i] $==$ s[j] then
			DP[i][j] = maxPalindromeSubsequence(s, DP, i+1, j-1) + 2
		else 
			DP[i][j] = max(maxPalindromeSubsequence(s, DP, i, j-1), maxPalindromeSubsequence(s, DP, i+1, j)) 
	return DP[i][j]
\end{lstlisting}
L'algoritmo proposto ha complessità $\mathcal{O}(n^{2})$ poichè la tabella che viene generata ha dimensione $n \times n$ è al limite verrà riempita tutta dall'algoritmo. La seconda richiesta del problema quella opzionale invece può essere soddisfatta utilizzando la tabella appena creata e modificando leggermente il codice iniziale in modo da far restituire.
\begin{lstlisting}[caption=Stampa Sottostringa Palindroma Massimale]
maxPalindromeSubsequence(string s, int n)
	int[][] DP = new int[1...n][1...n]
	for i = 1 to n do
		for j = 1 to n do
			DP[i][j] = -1
	maxPalindromeSubsequence(s, DP, 1, n)
	printMaxPalindromeSubsequence(s, DP, 1, n)
	
printMaxPalindromeSubsequence(string s, int[][] DP, int i, int j)
	if j $\geq$ i then 
		if s[i] $==$ s[j] then
			print(S[i])
				 printMaxPalindromeSubsequence(s, DP, i+1, j-1)
			print(S[i])
		else 
			if DP[i+1][j] > DP[i][j-1] then
				printMaxPalindromeSubsequence(s, DP, i+1, j)
			else
				printMaxPalindromeSubsequence(s, DP, i, j-1)
\end{lstlisting}
In questo caso la complessità dell'algoritmo è pari a $\mathcal{O}(n \cdot n^{2}) = \mathcal{O}(n^{3})$ poichè la stampa costa al massimo $2n$ (che si traduce in $\mathcal{O}(n)$): nel caso peggiore infatti verrà visitata al massimo tutta una riga e tutta una colonna (tutti i caratteri sono differenti).
\newpage
\end{document}