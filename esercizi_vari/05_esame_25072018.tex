\documentclass[../cheatSheetAlgoritmi.tex]{subfiles}
\begin{document}

\section{Esame 25/07/2018}
\textbf{Numero di modi in cui è possibile suddividere la stringa per ottenere la somma k} 

Avete dato un'occhiata al compito precedente? No? Male! 

Volevo dare un problema simile a \emph{Somma Massimale}, e per fare un esempio ho copiato e incollato una lista di numeri da un altro sito. 
Come a volte capita, gli spazi non stati copiati e questo è il risultato: \emph{"1151832175512"}.  

Non trovavo più il sito originale, mi ricordavo solo che la somma era pari a \emph{141}. Nel compito precedente, volevo stampare tutti i modi possibili in cui era possibile suddividere la stringa in una somma di numeri. Questa volta, voglio \emph{contare} tutti i modi possibili in cui è possibile suddividere l'input in una somma di numeri, ottenendo esattamente il valore che mi ricordo. 

Scrivere un algoritmo che prenda una stringa di numeri di lunghezza $n$ (rappresentata da un vettore di interi compresi fra $1$ e $9 - "0"$ escluso) e un intero $k \leq 100n$, e restituisca il numero di modi in cui è possibile rappresentare la stringa come sommatoria di interi positivi il cui valore è pari a $k$. Supponete di avere a disposizione una funzione \emph{value(V, i, j)} che  restituisce  il  valore  intero  corrispondente  alle  cifre comprese  fra  le  posizioni $i$ e $j$ in $V$. Ad  esempio, \emph{value(”54321”,2,4)} restituisce $432$.  
Questa funzione opera in tempo $\mathcal{O}(n)$; sarebbe possibile evitare questo costo, ma non importa. Discutere informalmente la correttezza della soluzione proposta e calcolare la complessità computazionale. 

\textbf{Soluzione} 

Per risolvere questo problema pensiamo ad una soluzione basata su programmazione dinamica. 

Immaginiamo che $DP[i][j]$ sia il numero di modi con cui è possibile ottenere j dal prefisso $V(i)$ (i primi i caratteri della stringa numerica). 

Abbiamo dunque la seguente formula ricorsiva: 
\begin{equation*}
  	DP[i][j] =\begin{cases}
    	1 & \text{$i = 0$ \textbf{and} $j = 0$}\\
    	0 & \text{$i = 0$ \textbf{and} $j > 0$}\\
    	0 & \text{$i > 0$ \textbf{and} $j \leq 0$}\\
    	\sum_{s = 1}^{i} DP[s - 1][r - value(V, s, i)] & \text{altrimenti}
  	\end{cases}
\end{equation*}
Possiamo dunque descrivere i seguenti casi: 
\begin{itemize}
		\item I modi possibili per ottenere il valore 0 dalle somme delle suddivisioni del prefisso della stringa numerica $V(i)$ nullo è 1, ovvero non prendere nulla
		\item Non c'è modo per ottenere il valore j se questo è maggiore di 0 e sto considerando un prefisso nullo oppure il prefisso non è nullo ma il valore da ottenere risulta essere negativo, questo perché i valori numerici all'interno della stringa $V$ sono tutti interi positivi
		\item Il caso ricorsivo è rappresentato dalla seguente condizione: supponiamo di conoscere già tutti i modi possibili per suddividere la stringa ed ottenere i diversi valori di j per tutti gli elementi $i' < i$ e $j' < j$. \\ Il numero di modi, per ottenere $j$ dati i primi $i$ elementi, è ottenuto dalla somma di tutti i possibili modi per ottenere $j$ aggregando più valori di $V$ precedenti ad esso, grazie alla funzione \emph{value} e considerando un elemento in meno: $\sum_{s = 1}^{i} DP[s - 1][r - value(V, s, i)]$
\end{itemize}
Il risultato che stiamo cercando si troverà nella posizione $DP[n][k]$. 

L'algoritmo proposto per la risoluzione del problema che sfrutta memoization è il seguente:
\newpage
\begin{lstlisting}[ caption = Conta i modi per formare la somma k]
int countSum(int[] V, int k)
	int[][] DP = new int[1...n][1...k]
	for i = 1 to n do
		for j = 1 to k do
			DP[i][j] = - 1
	return countSumRec(V, n, k, DP)

int countRec(int[] V, int i, int j, int[][] DP)
	if $i == 0$ and $j == 0$ then
		return 1
	else if $i == 0$ and $j > 0$ then
		return 0
	else if $j \leq 0$ then
		return 0
	else 
		if DP[i][j] < 0 then
			DP[i][j] = 0
			for s = 1 to i do
				DP[i][j] = DP[i][j] + countRec(V, s - 1, value(V, s, i), DP)
		return DP[i][j]
\end{lstlisting}
La tabella ha dimensione $n \cdot k$, dove $k = \mathcal{O}(n)$. \\ 
Per riempire ogni cella della tabella, sono necessari $\mathcal{O}(n^2)$ passi. \\ Quindi il costo totale è $\mathcal{O}(n^4)$. 
\newpage
\end{document}