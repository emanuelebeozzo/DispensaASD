\documentclass[../cheatSheetAlgoritmi.tex]{subfiles}
\begin{document}

\section{Esame 14/07/2015}
\textbf{Alice, Bob e Car: Nodo più vicino ad A e B senza passare da C} 

Si supponga che Alice, Bob e Carl siano posizionati su tre nodi in un grafo orientato $G= (V, E)$, associato ad una funzione di peso $w$ con pesi interi non negativi. Alice e Bob vogliono incontrarsi in un altro nodo del grafo, senza che Carl lo venga a sapere. 

Scrivere un algoritmo efficiente che sia in grado di individuare il nodo di incontro ottimale tra Alice e Bob, in modo tale che la somma dei cammini percorsi da entrambi (dalla loro posizione iniziale al nodo di incontro) sia minimizzata. 

L'algoritmo deve restituire tale somma minima. I due percorsi non devono passare per il nodo di Carl. Il punto di incontro non deve essere il nodo in cui risiede Carl. \\
Discutere informalmente la correttezza della soluzione proposta e calcolare la complessità computazionale. 

\textbf{Soluzione} 

Si può subito inture che la soluzione al problema è data dal nodo la cui somma di pesi per raggiungerlo da Alice e Bob sia la minore possibile, a patto che tale nodo non sia raggiunto passando da Carl. 

In altre parole, sia $A$ il vettore delle distanze minime la cui sorgente sia Alice e sia $B$ il vettore delle distanze minime la cui sorgente sia Bob, allora sia $i$ il nodo di incontro, deve valere che $A[i] + B[i] \leq A[j] + B[j]$ $ \forall j \in V$, dove A e B contengono le distanze minime non passando per Carl. 

Usufruiamo dunque di un algoritmo già visto a lezione, ovvero l'algoritmo per il calcolo dei cammini minimi a sorgente singola e lo facciamo partire sia da Alice che da Bob, non scordandoci di invalidare i cammini passanti per Carl; questo è semplice in quanto è sufficiente fare in modo che gli archi entranti nel nodo di Carl abbiano un peso pari a $+\infty$ facendo in modo che l'algoritmo chiamato non vada mai a selezionare tali percorsi se esiste il nodo di incontro che rispetta le caratteristiche fornite, nel caso non esistesse allora siamo consapevoli che se $\forall i \in V$ vale $A[i] + B[j] = +\infty$ e quindi tale nodo non esiste. 

L'algoritmo proposto è il seguente: 
\begin{lstlisting}[ caption = Nodo più vicino ad A e B senza passare da C]
int secretPlace(GRAPH G, NODE Alice, NODE Bob, NODE Carl)
	foreach $u \in G.V$ do
		G.w(u, Carl) = $+\infty$
	int[] A = camminiMinimi(G, Alice)
	int[] B = camminiMinimi(G, Bob)
	int min = $+ \infty$
	foreach $u \in G.V()$ do
		if A[u] + B[u] < min then
			min = A[u] + B[u]
	return min
\end{lstlisting}
La complessità di tale algoritmo dipende ovviamente dalla scelta della struttura dati di riferimento. Se siamo consapevoli di avere a disposizione un grafo molto denso utilizzeremo la procedura \emph{camminiMinimi} implementata grazie ad una coda a priorità basata su vettore (algoritmo di Dijkstra) e avremo una complessità pari a $\mathcal{O}(n^2)$. 

Se il grafo invece è sparso, o non denso, possiamo fare affidamento su una coda a priorità implementata tramite heap binario (algoritmo di Johnson) e dunque avere complessità pari a $\mathcal{O}(m \log n)$.
\newpage
\end{document}