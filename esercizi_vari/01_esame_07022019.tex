\documentclass[../cheatSheetAlgoritmi.tex]{subfiles}
\begin{document}

\section{Esami anni passati}
\subsection{Esame 07/02/2019}
\textbf{Esercizio A1} Trovare un limite superiore, il più stretto possibile per la seguente equazione di ricorrenza: \
\begin{equation*}
  	T(n)=\begin{cases}
    	T(\lfloor n/2 \rfloor) + T(\lfloor n/4 \rfloor)+ T(\lfloor n/8 \rfloor)+ 2T(\lfloor n/16 \rfloor) + 1 & \text{$n > 16$}\\
    	1 & \text{$n \leq 16$}
  	\end{cases}
\end{equation*} \\
\textbf{Soluzione A1} Non possiamo applicare il master theorem quindi proviamo a risolvere il problema per tentativi. Proviamo con il metodo di sostituzione a verificare e vale T(n) = $\mathcal{O}(n)$. Dimostriamo per induzione.\
\begin{itemize}
	\item Ipotesi induttiva: T(k) $\leq$ ck, per k $<$ n
	\item Passo induttivo:
\begin{equation*}
\begin{aligned}	
T(n)= T(\lfloor n/2 \rfloor) + T(\lfloor n/4 \rfloor)+ T(\lfloor n/8 \rfloor)+ 2T(\lfloor n/16 \rfloor) + 1\\
\text{$\leq$} c\lfloor n/2 \rfloor + c\lfloor n/4 \rfloor+ c\lfloor n/8 \rfloor+ 2c\lfloor n/16 \rfloor + 1\\ 
\text{$\leq$}  \dfrac{1}{2}cn + \dfrac{1}{4}cn + \dfrac{1}{8}cn + \dfrac{1}{8}cn + 1\\
=  cn + 1 \text{$\leq$} cn \\
\end{aligned}
\end{equation*}
L'ultima disequazione è falsa per un termine di ordine inferiore. 
\end{itemize}

Proviamo a dimostrare che 
\begin{equation*}
  \exists b \text{$>$} 0, \exists c \text{$>$} 0, \exists m \text{$\geq$} 0, : T(n) \text{$\leq$} cn-b, \forall n \text{$\geq$} m
\end{equation*}
\begin{itemize}
	\item Ipotesi induttiva: T(k) $\leq$ ck-b, per k $<$ n
	\item Passo induttivo:
	\begin{equation*}
		\begin{aligned}	
			T(n)= T(\lfloor n/2 \rfloor) + T(\lfloor n/4 \rfloor)+ T(\lfloor n/8 \rfloor)+ 2T(\lfloor n/16 \rfloor) + 1\\
			\text{$\leq$} c(lfloor n/2 \rfloor -b + c\lfloor n/4 \rfloor -b+ c\lfloor n/8 \rfloor -b + 2(c \lfloor n/16 \rfloor -b) + 1\\ 
			\text{$\leq$}  \dfrac{1}{2}cn -b + \dfrac{1}{4}cn -b + \dfrac{1}{8}cn -b + \dfrac{2}{16}cn -2b + 1\\
=  cn -5b + 1 \text{$\leq$} cn -b \\
		\end{aligned}
	\end{equation*}
	L'ultima disequazione è vera per ogni c e per b $\geq$ $\dfrac{1}{4}$. 
	\item Caso base: T(n)=1 $\leq$ cn-b, per tutti i valori di n compresi tra 1 e 16, ovvero: 
	\begin{equation*}
		c \text{$\geq$} \dfrac{b+1}{n}, \forall n: 	\text{$1 \leq n \leq 16$}
	\end{equation*}
	I valori $\dfrac{b+1}{n}$ sono minori o uguali di 5/4 (per T(1) vale 5/4), per 1 $\leq$ n $\leq$ 16; quindi tutte queste disequazioni sono soddisfatte da c $geq$ 5/4. 
\end{itemize}
Abbiamo quindi dimostrato che T(n)= $\mathcal{O}(n)$, con m= 1 e c $\geq$ 5/4.

\end{document}