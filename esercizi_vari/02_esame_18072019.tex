\documentclass[../cheatSheetAlgoritmi.tex]{subfiles}
\begin{document}

\section{Esame 18/07/2019}
\textbf{Esercizio B3}

Sia $A$ un vettore contenente n interi positivi.
\begin{itemize}
	\item Una sottosequenza crescente di $A$ è un sottosequenza di $A$ i cui valori sono ordinati.
	\item Il valore di una sottosequenza è pari alla somma dei suoi valori.
	\item Una sottosequenza crescente di $A$ è \emph{massimale} se ha valore massimo fra tutte le sottosequenze crescenti di $A$.
\end{itemize}
Scrivere un algoritmo 
\begin{center}
	\textbf{int} maxIncreasing(\textbf{int}[ ]$A$, \textbf{int} $n$)
\end{center}
che  restituisca  il valore di  una  sottosequenza  crescente massimale contenuta in $A$. Discutere  informalmente  la  correttezza  dell'algoritmo  e calcolare la sua complessità computazionale.
\textbf{Soluzione:}

Il seguente problema può essere risolto utilizzando l'algoritmo \emph{Longest Increasing Subsequence} poiché di fatto si riduce a prendere un numero e sommarlo ai precedenti solo se questo è maggiore di quelli nella sequenza.
\begin{lstlisting}[caption=Longest Increasing Subsequence]
int maxIncreasing(int[] A, int n)
   int[] DP = new int[1...n]
   	if n $\leq$ 1 then
   		DP[1] = A[1]
	for i = 2 to n do
		DP[i] = A[i]
		for j = 1 to i do
			if A[i] > A[j] and DP[i] $<$ DP[j] + A[i] then
				DP[i] = DP[j] + A[i]
	int maxValue = $-\infty$
	for i = 1 to n do
		maxValue = max(maxValue, DP[i])
	return iff(maxValue $\neq$ $-\infty$, maxValue, 0)
\end{lstlisting}
L'algoritmo può essere descritto mediante la seguente equazione di ricorrenza
\begin{equation*}
  	DP[i]=\begin{cases}
    	max_{1 \leq j < i}\{DP[j]\} + A[i] & \text{$A[i] > A[j]$ $\land$ $DP[i] < DP[j] + A[i]  $}\\
    	A[i] & \text{altrimenti}
  	\end{cases}
\end{equation*}
La complessità dell'algoritmo è dunque pari a $\mathcal{O}(n^{2})$ in quanto è necessario utilizzare due cicli per poter ottenere il valore della sottosequenza massima. Si sottolinea in questo caso l'importanza del fatto che la soluzione \textbf{non} necessariamente si troverà in $DP[n]$ bensì in una qualunque delle celle del vettore $DP$: basti pensare a una sequenza di numeri decrescente.

La correttezza può essere dimostrata provando che l'algoritmo possiede sottostruttura ottimale: sia $X_{P}$ il vettore contenente la sequenza di numeri di cui vogliamo calcolare la sottosequenza crescente massimale e sia $S$ l'insieme dei numeri la cui somma restituisce il valore di una delle sottosequenze crescenti massimale (contenente dunque la soluzione ottimale). Consideriamo ora il problema $X_{P} - \{j\}$, a cui di fatto viene tolto l'ultimo elemento: allora deve essere che la soluzione di tale problema è $S - \{j\}$, se così non fosse allora vorrebbe dire che esisterebbe un problema $S'$ che è soluzione ottima al problema, di conseguenza
\begin{center}
$\sum(S') > \sum(S - \{j\})$ $\rightarrow$ $\sum(S' + \{j\}) > \sum(S)$
\end{center}
ma ciò vorrebbe dire che il nostro insieme $S$ ha cardinalità minore di $S' + \{j\}$ e che dunque non conteneva un numero nella sottosequenza crescente ma ciò contraddice l'ipotesi che $S$ sia massimale e quindi sia la soluzione ottima.
 
\end{document}