\documentclass[../cheatSheetAlgoritmi.tex]{subfiles}
\begin{document}

\subsection{Esame 21/07/2014}
\textbf{Esercizio B4}\\
Un bambino scende una scala composta danscalini. Ad ogni passo, può decidere di fare $1,2,3,4$ scalini alla volta. Scrivere un algoritmo che determina in quanti modi diversi può scendere le scale. Ad esempio, se $n= 7$, alcuni dei modi possibili sono i seguenti (rappresentati dalla lunghezza dei passi in numero di scalini):
\begin{itemize}
	\item 1,1,1,1,1,1,1
	\item 1,2,4
	\item 4,2,1
	\item 2,2,2,1
	\item 1,2,2,1,1
\end{itemize}
Discutere informalmente la correttezza della soluzione proposta e calcolare la complessità computazionale\\
\textbf{Soluzione:}\\
Il problema può essere risolto mediante programmazione dinamica utilizzando una vettore $DP[i]$ come nel caso del problema del \emph{Domino} presente nella sezione di \emph{Programmazione Dinamica} della dispensa. Ricordiamo che la combinazioni di passi per i casi base sono:
\begin{itemize}
	\item n = 0, 1 combinazione
	\item n = 1, 1 combinazione 	[1]
	\item n = 2, 2 combinazioni	[1,1] [2]
	\item n = 3, 4 combinazioni	[1, 1, 1] [1, 2] [2, 1] [3]
\end{itemize}
Da n = 4 in poi è possibile utilizzare i casi base già presenti in quanto saranno pari alla somma dei precedenti 4 elementi del vettore. In generale dunque avremo che 
\begin{equation*}
  	DP[i]=\begin{cases}
  		1 & \text{$i = 0$}\\
  		0 & \text{$i < 0$}\\
  		\sum_{k = 1}^{4}{DP[i-k]} & \text{$altrimenti$}
  	\end{cases}
\end{equation*}
Questa può essere utilizzata per scrivere il seguente codice tramite programmazione dinamica
\begin{lstlisting}[caption=modi per scendere le scale]
int countStairs(int n)
	int[] DP = new int[0...n]
	DP[0] = 1
	for i = 1 to n do
		DP[i] = 0
		 for j = 1 to 4 do
		 	if i - j $\geq$ 0 then
		 		DP[i] = DP[i] + DP[i-j]
	return DP[n]
\end{lstlisting}
L'algoritmo proposto ha complessità $\mathcal{O}(n)$ in quanto il vettore verrà riempito completamente e ogni cella impiegherà al massimo un tempo costante per essere calcolata. Notare che $DP[n]$ è uguale al valore al valore del $(n+1)$-esimo Tetranacci Number.
\newpage
\end{document}